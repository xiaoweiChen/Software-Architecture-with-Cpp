
即使采取了必要的预防措施来确保依赖项和代码不受已知漏洞的影响,仍然存在一个可能危及安全策略的区域。所有应用程序都需要执行环境,这可能意味着容器、VM或操作系统。有时,这也意味着底层基础设施。

当运行应用程序的操作系统具有开放访问权限时,将应用程序加固到最大限度是不够的。这样,攻击者就可以直接从系统或基础设施级别获得对数据的未授权访问,而不是针对应用程序。

本节将重点介绍一些可以在最底层执行级别应用的加固技术。

\subsubsubsection{10.5.1\hspace{0.2cm}静态和动态链接}

链接是在编译后将所编写的代码及其各种依赖项(如标准库)组合在一起时发生的过程。链接可以发生在构建时、加载时(操作系统执行二进制文件时)或运行时,就像插件和其他动态依赖项一样。最后两个用例只适用于动态链接。

动态链接和静态链接有什么区别呢?使用静态链接,所有依赖项的内容都被复制到生成的二进制文件中。当程序加载时,操作系统将这个二进制文件放入内存中并执行它。静态链接由称为链接器的程序执行,作为构建过程的最后一步。

因为每个可执行文件都必须包含所有依赖项,所以静态链接的程序体积往往比较大。这也有好处,由于执行问题所需的所有东西都已经可用,因此执行速度会更快,并且将程序加载到内存中所花费的时间总是相同的。对依赖关系的任何更改都需要重新编译和链接,如果不改变产生的二进制文件,就无法升级依赖项。

动态链接中,生成的二进制文件包含编写的代码,但没有依赖项的内容,只有对实际库的引用,需要单独加载。在加载期间,动态加载器的任务是找到适当的库,并将它们与二进制文件一起加载到内存中。当多个应用程序同时运行,并且每个应用程序都使用类似的依赖项(例如JSON解析库或JPEG处理库)时,动态方式可以减少的二进制文件内存的使用,这是因为只能将给定库的副本加载到内存中。相比之下,对于静态链接的二进制文件,相同的库将作为生成的二进制文件的一部分反复加载。当需要升级某个依赖项时,可以这样做,而不需要接触系统的其他组件。下一次将应用程序加载到内存中时,将自动引用新升级的组件。

静态和动态链接也有安全问题,对动态链接的应用程序更容易获得未经授权的访问。这可以通过替换受损的动态库来代替常规的动态库,或者通过在每个新执行的进程中预加载某些库来实现。

当将静态链接与容器结合使用时(在后面的章节中会详细解释),将得到一个小型、安全的沙盒执行环境。甚至可以对基于微内核的VM使用这样的容器,从而减少攻击面。

\subsubsubsection{10.5.2\hspace{0.2cm}随机地址空间分配}

随机地址空间分配(ASLR)是一种用于防止基于内存的攻击的技术,工作原理是将程序和数据的标准内存布局替换为随机的布局。这意味着攻击者无法可靠地跳转到在没有ASLR的系统上存在的特定函数。

当与无执行(NX)位支持结合使用时,这种技术会更加有效。NX位将内存中的某些页面(如堆和堆栈)标记为只包含无法执行的数据。NX位支持已经在大多数主流操作系统中实现,只要硬件支持就可以使用。

\subsubsubsection{10.5.3\hspace{0.2cm}DevSecOps}

要在可预测的基础上交付软件增量,最好采用DevOps的理念。简而言之,DevOps意味着打破传统模式,鼓励商业、软件开发、软件运营、质量保证和客户之间的沟通。DevSecOps是DevOps的一种形式,强调在设计过程的每一步都要考虑到安全性。

这意味着正在构建的应用程序从一开始就具有可观察性,利用CI/CD流水,并定期扫描漏洞。DevSecOps让开发人员在底层基础设施的设计上有了发言权,也让操作专家在组成应用程序的软件包设计上有了发言权。由于每一个增量都代表一个工作系统(尽管不是完全功能),因此定期执行安全审计,因此比正常情况下花费的时间更少。这将导致更快、更安全的发布,并允许对安全事件做出更快的反应。









