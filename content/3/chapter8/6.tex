
下一章中,将关注持续集成和持续部署(CI/CD)。要使CI/CD流水正常工作,需要进行一组测试,以便在bug进入生产环境之前捕获它们。开发者和其团队需要确保所有业务需求都正确地在测试进行表达。

测试在几个层次上都很有用。对于行为驱动开发,业务需求构成了自动化测试的基础。但在构建的系统并不仅仅由业务需求组成,希望能够确保所有第三方集成都按预期工作,并且希望所有子组件(如微服务)可以相互接口。最后,希望正在构建的函数和类没有任何错误。

可以自动化的每个测试都是CI/CD流水的候选,它们中的每一个都在这个流水的某个地方有自己的位置。例如,端到端测试在作为验收测试部署之后最有意义。另一方面,单元测试在编译后直接执行时最有意义。毕竟,流水的目标是为了发现可能与规格不符的地方时,就立即中断。

每次运行CI/CD流水时,不必运行已经自动化的所有测试。如果每个流水的运行时间相对较短就更好了。理想情况下,应该在提交后几分钟内完成。如果想保持最小的运行时,如何确保一切都得到了适当的测试?

一种方法是为不同的目的准备不同的测试套件。例如,提交到特性分支的测试可以很少。由于每天都有很多人提交特性分支,这意味着它们只会简单地测试,而且结果会很快得到。将特性分支合并到共享开发分支需要大一些的测试用例集。这样,就可以确保没有破坏其他团队成员的任何东西。最后,将运行一组更广泛的案例,用于合并到生产分支。毕竟,即使测试需要相当长的时间,也希望对生产分支进行彻底的测试。

另一种是为CI/CD目的使用精简的测试用例集,并有一个额外的连续测试过程。此过程定期运行,并对特定环境的当前状态执行深入检查。这些测试可以包括安全性测试和性能测试,因此可以评估待提升环境的合格性。

当选择一个环境,并承认这个环境具备成为一个更成熟环境的所有特质时,就会出现提升,例如开发环境可以成为下一个阶段环境,或者阶段环境可以成为下一个生产环境。如果这种升级是自动发生的,那么提供自动回滚也是一个好做法,以防细微的差异(比如,域名或流量方面的差异)使新升级的环境不再通过测试。

这也提出了另一个重要的实践:始终在生产环境上运行测试。当然,这样的测试必须最小化干扰,但是它们应该告知,系统在给定的时间内都在正确地执行。

\subsubsubsection{8.6.1\hspace{0.2cm}测试的基础设施}

如果想将配置管理、基础设施作为代码或不可变部署的概念合并到应用程序的软件架构中,还应该考虑测试基础设施本身。可以使用工具来实现这一点,包括Serverspec、Testinfra、Goss和Terratest,这些都是比较主流的工具。

这些工具的适用范围略有不同,如下所述:

\begin{itemize}
\item 
Serverspec和Testinfra更专注于测试通过配置管理(如Salt、Ansible、Puppet和Chef)配置的服务器的实际状态,分别用Ruby和Python编写,插入到语言的测试引擎中。这意味着用于Serverspec的RSPec和用于Testinfra的Pytest。

\item 
Goss在范围和形式上都有点不同。除了测试服务器之外,还可以使用Goss来使用dgoss包装器测试项目中使用的容器。至于形式,没有使用在Serverspec或Testinfra中看到的命令式代码。相反,与Ansible或Salt类似,使用一个YAML文件来描述想要检查的状态。如果已经在使用声明式方法进行配置管理(例如,前面提到的Ansible或Salt),那么Goss可能更直观,因此更适合测试。

\item 
最后,Terratest是一个工具,允许测试基础设施的输出作为代码工具,如Packer和Terraform(因此得名)。就像Serverspec和Testinfra使用语言测试引擎为服务器编写测试一样,Terratest利用Go的测试包来编写适当的测试用例。
\end{itemize}

接下来,来了解一下如何使用这些工具来验证部署是否按照计划进行(至少从基础结构的角度来看)。

\subsubsubsection{8.6.2\hspace{0.2cm}Serverspec的测试}

下面是一个测试Serverspec的例子,检查Git在特定版本中的可用性和加密配置文件:

\begin{tcblisting}{commandshell={}}
# We want to have git 1:2.1.4 installed if we're running Debian
describe package('git'), :if => os[:family] == 'debian' do
  it { should be_installed.with_version('1:2.1.4') }
end
# We want the file /etc/letsencrypt/config/example.com.conf to:
describe file('/etc/letsencrypt/config/example.com.conf') do
  it { should be_file } # be a regular file
\end{tcblisting}
\begin{tcblisting}{commandshell={}}
  it { should be_owned_by 'letsencrypt' } # owned by the letsencrypt user
  it { should be_mode 600 } # access mode 0600
  it { should contain('example.com') } # contain the text example.com
                                     # in the content
end
\end{tcblisting}

Ruby DSL语法应该是可读的,即使对那些不日常使用Ruby的人也是如此。

\subsubsubsection{8.6.3\hspace{0.2cm}Testinfra的测试}

下面是Testinfra的一个测试示例,检查Git在特定版本中的可用性和加密配置文件:

\begin{lstlisting}[style=stylePython]
# We want Git installed on our host
def test_git_is_installed(host):
	git = host.package("git")
	# we test if the package is installed
	assert git.is_installed
	# and if it matches version 1:2.1.4 (using Debian versioning)
	assert git.version.startswith("1:2.1.4")

# We want the file /etc/letsencrypt/config/example.com.conf to:
def test_letsencrypt_file(host):
	le = host.file("/etc/letsencrypt/config/example.com.conf")
	assert le.user == "letsencrypt" # be owned by the letsencrypt user
	assert le.mode == 0o600 # access mode 0600
	assert le.contains("example.com") # contain the text example.com in the contents
\end{lstlisting}

Testinfra使用纯Python语法。可读性不错,但就像Serverspec一样,可能需要一段时间的了解,才能自信地在其中编写测试。

\subsubsubsection{8.6.4\hspace{0.2cm}Goss的测试}

下面是Goss的YAML文件的一个例子,检查Git在特定版本的可用性和加密配置文件:

\begin{tcblisting}{commandshell={}}
# We want Git installed on our host
package:
  git:
    installed: true # we test if the package is installed
  versions:
  - 1:2.1.4 # and if it matches version 1:2.1.4 (using Debian versioning)
\end{tcblisting}
\begin{tcblisting}{commandshell={}}
file:
  # We want the file /etc/letsencrypt/config/example.com.conf to:
  /etc/letsencrypt/config/example.com.conf:
    exists: true
  filetype: file # be a regular file
  owner: letsencrypt # be owned by the letsencrypt user
  mode: "0600" # access mode 0600
  contains:
  - "example.com" # contain the text example.com in the contents
\end{tcblisting}

读和写YAML的语法可能需要最少的准备。但是,如果项目已经使用了Ruby或Python,那么在编写更复杂的测试时,可以坚持使用Serverspec或Testinfra。









