
区分不同类型的测试和了解特定的测试框架(或几种)是不够的。开始测试时,就会注意到,并不是所有的类都可以轻松地测试。有时,可能会觉得需要访问私有属性或方法。如果想保持良好的架构原则,请抵制这种冲动!相反,可以考虑测试通过类型的公共API可用的业务需求,或者重构类型,以便有另一个代码单元可以测试。

\subsubsubsection{8.5.1\hspace{0.2cm}测试和类设计——冲突}

可能面临的问题不是测试框架不充分,而是遇到的是设计不当的类。即使类的行为和外观可能是正确的,除非允许测试,否则其设计不正确。

然而,这是个好消息。说明可以在不方便这样做之前修复问题。类设计可能会在稍后,基于它构建类层次结构时再次困扰开发者。在测试执行过程中修改设计可以减少可能的技术债。

\subsubsubsection{8.5.3\hspace{0.2cm}防御性编程}

防御性编程并不是一种安全特性,其名字来自于保护类和函数的使用不违背它们的初衷。它与测试没有直接关系,但它是一种很好的设计模式,因为它提高了代码的质量,使项目具有前瞻性。

防御性编程从静态类型开始。若创建一个函数来处理自定义类型作为参数,必须确保没有人会用一些意外的值来使用它。用户必须有意识地检查函数的期望并准备相应的输入。

C++中可以在编写模板代码时利用类型安全特性。当为客户的评论创建一个容器时,可以接受任何类型的列表,并从中复制。为了得到更详细的错误信息和精心设计的检查,可以这样做:

\begin{lstlisting}[style=styleCXX]
class CustomerReviewStore : public i_customer_review_store {
public:
	CustomerReviewStore() = default;
	explicit CustomerReviewStore(const std::ranges::range auto
	&initial_reviews) {
		static_assert(is_range_of_reviews_v<decltype(initial_reviews)>,
			"Must pass in a collection of reviews");
		std::ranges::copy(begin(initial_reviews), end(initial_reviews),
			begin(reviews_));
	}
// ...
private:
	std::vector<review> reviews_;
};
\end{lstlisting}

\texttt{explicit}关键字保护不需要隐式类型转换。通过指定输入参数满足范围概念,可以确保只使用有效的容器进行编译。由于使用了概念,可以让防御代码输出更清晰的错误消息。在代码中使用\texttt{static\_assert}也是一个很好的防御措施,因为它允许在需要的时候提供错误消息。\texttt{is\_range\_of\_reviews}的检查可以这样实现:

\begin{lstlisting}[style=styleCXX]
template <typename T>
	constexpr bool is_range_of_reviews_v =
		std::is_same_v<std::ranges::range_value_t<T>, review>;
\end{lstlisting}

通过这种方式,可以确保获得的范围包含合适类型的“评论”。

静态类型不会在运行时阻止无效的值传递给函数,这就是为什么防御型编程的下一种形式是检查的前提条件。这样,若有出现问题的迹象,代码就会失败,这比返回一个传播到系统其他部分的无效值要好得多。C++中有契约之前,可以使用前面章节提到的GSL库来检查代码的前置和后置条件:

\begin{lstlisting}[style=styleCXX]
void post_review(review review) final {
	Expects(review.merchant);
	Expects(review.customer);
	Ensures(!reviews_.empty());
	
	reviews_.push_back(std::move(review));
}
\end{lstlisting}

这里,通过使用\texttt{Expects}宏,检查传入的评论是否确实具有商家和评论人的ID集。除了不会发生的情况外,还可以防止在使用“确保后置条件宏”时,向存储添加复查失败的情况。

当谈到运行时检查时,首先想到的是检查一个或多个属性是否不是\texttt{nullptr}。避免这个问题的最好方法是区分可空资源(那些可以接受空指针的资源)和非空资源。C++17的标准库中,\texttt{std::optional}是 一个很棒的工具。如果可以,请在设计API时使用它。

\subsubsubsection{8.5.3\hspace{0.2cm}无聊的重复——先写测试}

这句话已经重复过很多次了,但是很多人会“忘记”。当实际编写测试时,必须做的第一件事是减少创建难以测试的类的风险。从API使用开始,需要调整实现以最佳地服务于API。这样,就会得到更易于使用和测试的API。当在实现测试驱动开发(TDD)或在编写代码之前编写测试时,还可以将最终实现的依赖进行注入,这意味着可以创建更松散地耦合类。

用另一种方法(先编写类,然后再向类添加单元测试)会让编写的代码更容易,但更难以测试。当测试变得困难时,就会有跳过它的冲动。













