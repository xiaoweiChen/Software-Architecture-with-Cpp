
Kubernetes是一个可扩展的开源平台,用于自动化和管理容器应用程序。它有时称为k8s,因为它以“k”开头,以“s”结尾,中间有八个字母。

设计基于Google内部使用的Borg系统。Kubernetes的一些特征如下:

\begin{itemize}
\item 
自动扩展的应用程序

\item 
可配置的网络

\item 
批处理执行

\item 
应用统一升级

\item 
具有运行高可用性应用程序的能力

\item 
声明式配置
\end{itemize}

在团队中有不同的方式来管理Kubernetes,选择其中一种需要分析与之相关的成本和收益。

\subsubsubsection{15.3.1\hspace{0.2cm}Kubernetes的结构}

虽然可以在一台机器上运行Kubernetes(例如,使用minikube、k3s或k3d),但不建议在生产环境中这样做。单机集群的功能有限,而且没有故障转移机制。Kubernetes集群通常的大小是六台或更多机器。然后,三个机器组成控制台。其他三个是工作节点。

3台机器的最低要求来自于这样一个事实,这是提供高可用性的最低数量。控制台节点也可以作为工作节点使用,但不推荐这样做。

\hspace*{\fill} \\ %插入空行
\noindent
\textbf{控制台}

在Kubernetes中,很少与单个工作节点交互。相反,所有API请求都进入控制台。然后控制台根据请求决定采取的操作,然后它与工作节点通信。

与控制台的交互可以采取的几种形式:

\begin{itemize}
\item 
通过kubectl命令行

\item 
使用Web仪表盘

\item 
在kubectl之外的应用程序中使用Kubernetes API
\end{itemize}

控制台节点通常运行API服务器、调度器、配置存储(etcd),可能还会运行一些进程来处理特定的需求。例如,Kubernetes集群部署在公有云(如Google云平台)上,在控制平面节点上运行云控制器。云控制器与云提供商的API交互,以替换故障机器、提供负载平衡器或分配外部IP地址。

\hspace*{\fill} \\ %插入空行
\noindent
\textbf{工作节点}

构成控制台和工作池的节点是工作负载将在其上运行的实际机器,可能是地托管的物理服务器、私有托管的虚拟机或来自云提供商的虚拟机。

集群中的每个节点至少运行下面三个程序:

\begin{itemize}
\item 
容器运行时(例如,Docker引擎或cri-o),允许机器处理应用程序容器

\item 
kubelet,负责接收来自控制台的请求,并基于这些请求管理各个容器

\item 
kube-proxy,负责节点级的网络和负载平衡
\end{itemize}

\subsubsubsection{15.3.2\hspace{0.2cm}部署Kubernetes的方法}

正如上一节中了解到的,部署Kubernetes有多种不同的方法。

其中之一是将其部署到本地托管的裸服务器上。这样的好处是,对于大型应用程序来说,这可能比云提供商提供的服务更便宜。这种方法有一个缺点——需要在必要时提供额外的节点。

为了缓解这个问题,可以在裸服务器上运行一个虚拟化设备。这使得使用Kubernetes内置的云控制器自动提供必要的资源成为可能。仍可以对成本进行控制,但手工工作减少了。虚拟化会增加一些开销,但在大多数情况下,这是一种权衡。

如果对自托管服务器不感兴趣,可以将Kubernetes部署在云提供商提供的虚拟机上运行。通过选择这种方式,可以使用一些现有模板进行最佳设置。有terrform和Ansible模块可以在流行的云平台上构建集群。

最后,还有来自主要云计算参与者的托管服务。只需要支付其中一些工作节点的费用,而控制台是免费的。

当在公共云中运行时,为什么要选择自托管的Kubernetes,而不是托管的服务呢?其中一个原因是需要特定版本的Kubernetes。云提供商在引入更新时通常会有延迟。

\subsubsubsection{15.3.3\hspace{0.2cm}理解Kubernetes的概念}

Kubernetes介绍了一些概念,如果第一次听到它们,可能会感到陌生或困惑。当了解他们的目的时,应该更容易理解Kubernetes的特别之处。以下是Kubernetes最常见的一些概念:

\begin{itemize}
\item 
容器,特别是应用程序容器,是分发和运行单个应用程序的一种方法。可以运行未修改的应用程序所需的代码和配置。

\item 
Pod是Kubernetes的基本构件,是原子的,由一个或多个容器组成。Pod中的所有容器共享相同的网络接口、卷(如持久存储或机密)和资源(CPU和内存)。

\item 
部署是描述工作负载,及其生命周期特性的高级对象。通常管理一组Pod副本,允许滚动升级,并在失败时管理回滚。这使得扩展和管理Kubernetes应用程序的生命周期变得很容易。

\item 
DaemonSet是一个类似于部署的控制器,它管理Pod分布的位置。虽然部署关注的是保持给定数量的副本,但daemonset将Pod分散到所有工作节点。主要用在每个节点上运行系统级服务,例如监视或日志记录代理。

\item 
作业是为一次性任务而设计的,部署中的Pod在容器终止时自动重启。它们适用于所有在网络端口上监听请求的alwayson服务。但是,部署不适合批处理作业,例如缩略图生成,通常只希望在需要时运行这些作业。作业创造一个或多个Pod,并观察他们,直到他们完成指定的任务。当成功终止特定数量的Pod时,作业认定为完成。

\item 
CronJobs,是指在集群中周期性运行的作业。

\item 
服务表示在集群中执行的特定功能,有一个与它们相关联的网络端点(通常是负载均衡的)。服务可以由一个或多个Pod执行,服务的生命周期独立于许多Pod的生命周期。由于Pod是短暂的,可能在任何时候创建和销毁。服务将各个Pod抽象出来,以实现高可用性。为了方便使用,服务有自己的IP地址和DNS名称。
\end{itemize}

\hspace*{\fill} \\ %插入空行
\noindent
\textbf{声明式方法}

已经在第9章中讨论了声明式方法和命令式方法之间的区别。Kubernetes采用声明式方法。提供描述集群所需状态的资源,而不是给出关于需要采取的步骤的指示。由控制平面来分配内部资源,以满足需求。

可以直接使用命令行添加资源。这可以快速进行测试,可能希望在大多数情况下都能跟踪创建的资源。因此,大多数人使用清单文件,提供所需资源的编码描述。清单通常是YAML文件,也可以使用JSON。

下面是一个带有单个Pod的YAML清单示例:

\begin{tcblisting}{commandshell={}}
apiVersion: v1
kind: Pod
metadata:
  name: simple-server
  labels:
    app: dominican-front
\end{tcblisting}
\begin{tcblisting}{commandshell={}}
spec:
  containers:
    - name: webserver
      image: nginx
      ports:
        - name: http
          containerPort: 80
          protocol: TCP
\end{tcblisting}

第一行是强制性的,在清单中将使用哪个API版本。有些资源仅在扩展中可用,因此这是解析器关于如何操作的信息。 

第二行描述创建的资源。接下来是元数据和资源规范。

名称在元数据中是必需的,因为这是区分一个资源与另一个资源的方法。如果想创建另一个具有相同名称的Pod,将得到一个错误,说明这样的资源已经存在。标签是可选的,在编写选择器时很有用。例如,想创建一个允许连接到Pod的服务,将使用一个值等于dominican-front的标签匹配应用的选择器。

规范也是必需的部分,因为它描述了资源的实际内容。示例中,列出了在Pod中运行的所有容器。准确地说,一个名为webserver的容器使用了一个来自Docker Hub的nginx镜像。因为想从外部连接到Nginx web服务器,所以也公开了服务器正在监听的容器端口80。端口描述中的名称为可选参数。

\subsubsubsection{15.3.4\hspace{0.2cm}Kubernetes网络}

Kubernetes支持可插拔网络架构,有几种驱动程序可以根据需求使用。无论选择哪种驱动程序,有些概念是通用的。以下是常规的组网场景。

\hspace*{\fill} \\ %插入空行
\noindent
\textbf{容器对容器的通信}

一个Pod可以容纳几个不同的容器。由于网络接口绑定到Pod,而不是容器,因此每个容器都在相同的网络名称空间中操作。这意味着各种容器可以使用本地主机网络彼此寻址。

\hspace*{\fill} \\ %插入空行
\noindent
\textbf{Pod对Pod的通信}

每个Pod都分配一个内部集群本地IP地址。当Pod删除了,地址就不存在了。当一个Pod知道另一个的地址时,因为它们共享相同的网络,所以可以连接到另一个暴露的端口。对于这个通信模型,可以将Pod看作是托管容器的VM。这种方法很少使用,因为首选的方法是Pod到服务的通信。

\hspace*{\fill} \\ %插入空行
\noindent
\textbf{Pod对服务的通信}

Pod对服务通信是集群内最流行的通信用例。每个服务都有一个单独的IP地址和分配给它的DNS。当一个Pod连接到服务时,该连接被代理到由服务选择的组中的一个Pod。代理是前面描述的kube-proxy工具的任务。

\hspace*{\fill} \\ %插入空行
\noindent
\textbf{外部对内部的通信}

外部流量通常通过负载均衡器到达集群。它们要么定到特定的服务或入口控制器,要么使用特定的服务或入口控制器处理。当外部公开的服务处理通信时,行为类似于Pod到服务的通信。有了入口控制器,就有了其他可用的特性,允许路由、可观察性或高级负载平衡。

\subsubsubsection{15.3.5\hspace{0.2cm}何时使用Kubernetes}

在组织中引入Kubernetes需要一些成本。Kubernetes提供了许多好处,例如自动可扩展性、自动化或部署场景。然而,这些好处可能不足以证明必要的成本是合理的。

这项投资成本涉及几个领域:

\begin{itemize}
\item 
基础设施成本:与运行控制台和工作节点相关的成本可能相对较高。此外,若希望使用各种Kubernetes扩展,例如GitOps或服务网格(稍后将介绍),那么成本可能会增加。还需要额外的资源来运行,并在应用程序的常规服务之上提供更多的开销。除了节点本身,还应该考虑其他成本。Kubernetes的一些特性在部署到受支持的云提供商时工作得最好。这意味着为了从这些功能中受益,必须遵循以下方式:

\begin{enumerate}[label=\alph*]
\item 
将工作负载转移到特定支持的云上

\item
为选择的云提供商,实现自己的驱动程序

\item
将基础设施迁移到一个虚拟化的支持API的环境,如VMware vSphere或OpenStack。
\end{enumerate}

\item 
运营成本:Kubernetes集群和相关服务需要维护。尽管应用程序的维护减少了,但保持集群运行的成本略微抵消了这一好处。

\item 
学习成本:整个产品团队必须学习新的概念。即使有一个专业的平台团队,为开发人员提供易于使用的工具,开发人员仍然需要了解所做的工作如何影响整个系统,以及应该使用哪些API。
\end{itemize}

引入Kubernetes之前,首先考虑是否能够负担它所需的成本。












