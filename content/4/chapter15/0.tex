
As the name suggests, cloud-native design describes the application's architecture built, first and foremost, to operate in the cloud. It is not defined by a single technology or language, but rather takes advantage of all that the modern cloud platforms offer. 

This may mean a combination of using Platform-as-a-Service (PaaS) whenever necessary, multi-cloud deployments, edge computing, Function-as-a-Service (FaaS), static file hosting, microservices, and managed services. It transcends the boundaries of traditional operating systems. Instead of targeting the POSIX API and UNIX-like operating systems, cloud-native developers build on higher-level concepts using libraries and frameworks such as boto3, Pulumi, or Kubernetes.


本章将讨论以下内容:

\begin{itemize}
\item 
Understanding cloud-native 

\item 
Using Kubernetes to orchestrate cloud-native workloads

\item 
Connecting services with a service mesh

\item 
Observability in distributed systems

\item 
Going GitOps
\end{itemize}

By the end of the chapter, you'll have a good understanding of how modern trends in software architecture can be used in your applications.