
面向服务的架构是一个软件设计的例子,特点是组件松散耦合,这些组件相互提供服务。这些组件通常通过网络使用共享通信协议,服务意味着可以在原始组件外部访问的功能单元。组件的一个例子可以是映射服务,该服务根据地理坐标提供该地区的地图。

根据定义,服务有四个属性:

\begin{itemize}
\item 
具有已定义结果的业务活动的表示。

\item 
自包含的。

\item 
对用户黑盒。

\item 
由其他服务组成。
\end{itemize}

\subsubsubsection{12.2.1\hspace{0.2cm}实现方法}

面向服务的架构没有规定如何实现面向服务,这个术语可以应用于许多不同的实现。对于某些方法是否应该视为面向服务的架构,存在一些讨论。我们不想参与这些讨论,只是想强调一些常称为SOA的方法。

\hspace*{\fill} \\ %插入空行
\noindent
\textbf{企业服务总线}

当有人说到面向服务的架构时,ESB通常是第一个关联。它是实现SOA的最古老的方法之一。

ESB借鉴了计算机硬件体系结构,硬件结构采用PCI等计算机总线实现模块化。这样,只要每个人都遵守总线要求的标准,第三方提供商就能够独立于主板制造商实现模块(如图形卡、声卡或I/O接口)。

与PCI非常相似,ESB体系结构旨在构建一种标准的、通用的方式,以允许松散耦合的服务进行交互。这些服务预计将独立开发和部署,还可以组合为异构服务。

与SOA本身一样,ESB不是由全局标准定义的。要实现ESB,需要在系统中建立一个组件,这个组件就是总线本身。ESB上的通信是事件驱动的,通常通过面向消息的中间件和消息队列来实现。

企业服务总线组件服务于以下角色:

\begin{itemize}
\item 
控制服务的部署和版本控制

\item 
维护服务的冗余

\item 
服务之间的消息路由

\item 
监视和控制消息交换

\item 
解决组件之间的争用

\item 
提供公共服务,如事件处理、加密或消息入队

\item 
加强服务素质(QOS)
\end{itemize}

实现企业服务总线功能的产品既有专有的商业产品,也有开源的产品。以下是一些最受欢迎的开源产品:

\begin{itemize}
\item 
Apache Camel

\item 
Apache ServiceMix

\item 
Apache Synapse

\item 
JBoss ESB

\item 
OpenESB

\item 
Red Hat Fuse(基于Apache Camel)

\item 
Spring Integration
\end{itemize}

最受欢迎的商业产品有:

\begin{itemize}
\item 
IBM Integration Bus (已经被IBM WebSphere ESB替代)

\item 
Microsoft Azure Service Bus

\item 
Microsoft BizTalk Server

\item 
Oracle Enterprise Service Bus

\item 
SAP Process Integration
\end{itemize}

与本书中介绍的所有模式和产品一样,在决定使用特定的架构之前,必须考虑其优点和缺点。引入企业服务总线的一些好处如下:

\begin{itemize}
\item 
更好的服务扩展性

\item 
分布式负载

\item 
可以专注于配置,而不是实现服务中的自定义集成

\item 
设计松散耦合服务的简单方法

\item 
服务可替换

\item 
内置的冗余功能
\end{itemize}

另一方面,缺点主有以下几个方面:

\begin{itemize}
\item 
单点故障——ESB组件的故障意味着整个系统的中断。

\item 
配置比较复杂,影响维护。

\item 
消息队列、消息转换和ESB提供的其他服务可能会降低性能,甚至成为瓶颈。
\end{itemize}

\hspace*{\fill} \\ %插入空行
\noindent
\textbf{Web服务}

Web服务是面向服务架构的另一种流行实现。根据Web服务的定义,Web服务是一台机器向另一台机器(或操作员)提供的服务,其中通过万维网协议进行通信。尽管万维网管理机构W3C允许使用其他协议,如FTP或SMTP,但Web服务通常使用HTTP作为传输协议。

虽然可以使用专有的解决方案来实现Web服务,但大多数实现都基于开放协议和标准。尽管许多方法通常称为Web服务,但彼此之间有着本质上的不同。本章的后面,将详细描述各种方法。现在,让我们关注它们的共同特性。

\hspace*{\fill} \\ %插入空行
\noindent
\textbf{Web服务的优缺点}

Web服务的好处如下:

\begin{itemize}
\item 
使用流行的Web标准

\item 
很多工具

\item 
可扩展性
\end{itemize}

下面是缺点:

\begin{itemize}
\item 
开销很大

\item 
有些实现复杂(例如,SOAP/WSDL/UDDI规范)
\end{itemize}

\hspace*{\fill} \\ %插入空行
\noindent
\textbf{消息和流}

介绍企业服务总线体系结构时,已经提到了消息队列和消息代理。除了作为ESB实现的一部分外,消息传递系统还可以是独立的架构元素。

\hspace*{\fill} \\ %插入空行
\noindent
\textbf{消息队列}

消息队列是用于进程间通信(IPC)的组件。顾名思义,它们使用队列数据结构在不同进程之间传递消息。通常,消息队列是面向消息中间件(MOM)设计的一部分。

在最低级别,消息队列在UNIX规范中可用,在System V和POSIX中都有。虽然在单台机器上实现IPC很有趣,但我们更希望关注适合于分布式计算的消息队列。

目前,开源软件中与消息队列相关的标准有三种:

\begin{enumerate}
\item 
高级消息队列协议(AMQP),一种运行在7层OSI模型的应用层上的二进制协议。流行的实现包括以下几点:

\begin{itemize}
\item 
Apache Qpid

\item 
Apache ActiveMQ

\item 
RabbitMQ

\item 
Azure Event Hubs

\item 
Azure Service Bus
\end{itemize}

\item 
流式面向文本的消息传递协议(STOMP),类似于HTTP的基于文本的协议(使用诸如CONNECT、SEND、SUBSCRIBE等动词)。流行的实现包括以下几点:

\begin{itemize}
\item 
Apache ActiveMQ

\item 
RabbitMQ

\item 
syslog-ng
\end{itemize}

\item 
MQTT,一种针对嵌入式设备的轻量级协议。流行的实现包括家庭自动化解决方案,如以下:

\begin{itemize}
\item 
OpenHAB

\item 
Adafruit IO

\item 
IoT Guru

\item 
Node-RED

\item 
Home Assistant

\item 
Pimatic

\item 
AWS IoT

\item 
Azure IoT Hub
\end{itemize}

\end{enumerate}

\hspace*{\fill} \\ %插入空行
\noindent
\textbf{消息代理}

消息代理处理消息系统中消息的转译、验证和路由。与消息队列一样,也是MOM的一部分。

通过使用消息代理,可以最大限度地减少应用程序对系统其他部分的感知,这会让系统的设计变得松散耦合,因为消息代理承担了与消息上的通用操作相关的所有负担,从而称为PublishSubscribe (PubSub)设计模式。

代理通常为接收者管理消息队列,也能够执行其他功能,例如:

\begin{itemize}
\item 
将消息从一种表示转换为另一种表示

\item 
验证消息发送方、接收方或内容

\item 
将消息路由到一个或多个目的地

\item 
聚合、分解和重组传输中的消息

\item 
从外部服务检索数据

\item 
通过与外部服务的交互来增强和丰富消息

\item 
处理和响应错误和其他事件

\item 
提供不同的路由模式,如PubSub
\end{itemize}

消息代理的主流实现如下:

\begin{itemize}
\item 
Apache ActiveMQ

\item 
Apache Kafka

\item 
Apache Qpid

\item 
Eclipse Mosquitto MQTT Broker

\item 
NATS

\item 
RabbitMQ

\item 
Redis

\item 
AWS ActiveMQ

\item 
AWS Kinesis

\item 
Azure Service Bus
\end{itemize}

\hspace*{\fill} \\ %插入空行
\noindent
\textbf{云计算}

云计算是一个宽泛的术语,有很多不同的含义。最初,术语云指的是架构不应该太担心的抽象层。例如,这可能意味着由专门的运营团队管理服务器和网络基础设施。后来,服务提供商开始将术语云计算应用到他们自己的产品中,这些产品抽象了底层基础设施的复杂性。不必单独配置每个基础设施,而是可以使用简单的应用程序编程接口(Application Programming Interface, API)来设置所有必要的资源。

如今,云计算已经发展到包括许多应用程序架构的新方法,可能包括以下内容:

\begin{itemize}
\item 
托管服务,如数据库、缓存层和消息队列

\item 
可扩展的工作负载编排

\item 
容器部署和编制平台

\item 
无服务的计算平台
\end{itemize}

考虑使用云时,要记住的一点是,在云中托管应用程序需要专门为云设计的架构。通常,还意味着专门为给定的云提供商设计的架构。

这意味着选择云提供商,不仅仅是在给定的时间点上决定一个选择是否比另一个更好,更换供应商的未来成本可能太大,不值得这样做。提供商之间的迁移需要更改架构,而对于一个没问题的应用程序,这些更改可能会超过迁移所节省的成本。

云架构设计的另一个后果。对于遗留应用程序,这意味着为了利用云的优势,应用程序首先必须重新构建和编写。迁移到云不仅仅是将二进制文件和配置文件从本地托管复制到云提供商管理的虚拟机。这种方法只会浪费金钱,因为只有当应用程序具有可扩展性和云感知能力时,云计算才具有成本效益。

云计算并不一定意味着使用外部服务和从第三方提供商租赁机器,还有一些解决方案,比如在本地运行的OpenStack,可以使用已经拥有的服务器来获得云计算的红利。

我们将在本章后面讨论托管服务。容器、云本地设计和无服务器架构将在本书后面的章节中具体描述。

\hspace*{\fill} \\ %插入空行
\noindent
\textbf{微服务}

关于微服务是否是SOA的一部分存在一些争论。大多数时候,术语SOA几乎等同于ESB设计。微服务在许多方面与ESB相反。这导致人们认为微服务是不同于SOA的一种模式,是软件架构发展的下一步。

我们相信它们是一种现代的SOA方法,旨在消除ESB中出现的一些问题。毕竟,微服务非常符合面向服务架构的定义。

\subsubsubsection{12.2.2\hspace{0.2cm}面向服务架构的优点}

将系统的功能拆分到多个服务上有几个好处。首先,每个服务都可以单独维护和部署。这有助于团队专注于给定的任务,而不需要理解系统内的每一个可能的交互,其还支持敏捷开发,因为测试只需要覆盖特定的服务,而不是整个系统。

第二个好处是服务的模块化有助于创建分布式系统。使用网络(通常基于Internet协议)作为通信手段,服务可以在不同的机器之间分割,以提供扩展性、冗余和更好的资源使用。

当每个服务都有许多生产者和消费者时,实现新特性和维护现有软件是一项困难的任务。这就是SOA鼓励使用文档化和版本化API的原因。

另一种使服务生产者和消费者更容易交互的方法是,使用描述如何在不同服务之间传递数据和元数据的既定协议。这些协议可能包括SOAP、REST或gRPC。

API和标准协议的使用使得创建新服务变得很容易,这些新服务提供了比现有服务更高的价值。考虑到我们有一个服务A,它返回地理位置,另一个服务B,它提供给定位置的当前温度,可以调用A并在向B的请求中使用它的响应。这样,就可以获得当前位置的当前温度,而不用自己实现整个逻辑。

这两个服务的所有复杂性和实现细节对外都是未知的,我们将它们视为黑盒。这两个服务的维护者也可能引入新的功能并发布服务的新版本,但不需要对外通知。

使用面向服务的架构进行测试和试验也比使用单块应用程序更容易,小更改不需要重新编译整个代码库。通常可以使用客户端工具以特定的方式调用服务。

回到天气和地理位置服务的示例。如果两个服务都公开一个REST API,就能够构建一个原型,只使用cURL客户机手动发送适当的请求。当确认响应是令人满意的时,可以开始编写代码,使整个操作自动化,并可能将结果作为另一个服务公开。

\begin{tcolorbox}[colback=blue!5!white,colframe=blue!75!black, title=Note]
\hspace*{0.7cm}要获得SOA的好处,需要记住所有服务都必须是松散耦合的。如果服务依赖于彼此的实现,这意味着它们不再是松散耦合的,而是紧密耦合的。理想情况下,给定的服务都可以被不同的类似服务替代,而不会影响整个系统的运行。
\end{tcolorbox}

在天气和位置的例子中,这意味着用另一种语言重新实现位置服务(比如从Go切换到C++)应该不会影响到该服务的下游使用者,只要他们使用已建立的API即可。

通过发布新的API版本,仍然有可能在API中引入破坏性的变化。连接到1.0版本的客户机将观察遗留行为,而连接到2.0版本的客户机将受益于bug修复、更好的性能和其他以兼容性为代价的改进。

对于依赖于HTTP的服务,API版本控制通常发生在URI级别。因此,当调用\url{URLhttps://service.local/v1/customer}时可以访问版本1.0、1.1或1.2的API,而版本2.0的API位于\url{https://service.local/v2/customer}。然后,API网关、HTTP代理或负载均衡器能够将请求切换到适当的服务。

\subsubsubsection{12.2.3\hspace{0.2cm}SOA的挑战}

引入抽象层总是要付出代价的。同样的规则也适用于面向服务的架构。在查看企业服务总线、Web服务或消息队列和代理时,很容易看到抽象成本。可能不那么明显的是,微服务也是有代价的。其成本与它们使用的远程过程调用(Remote Procedure Call, RPC)框架有关,以及与服务冗余和功能重复相关的资源消耗。 

与SOA相关的另一个批评是缺乏统一的测试框架,开发应用程序服务的单个团队可能会使用其他团队不知道的工具。与测试相关的其他问题是,组件的异构性质和互换性意味着需要测试大量的组合。一些组合可能会引入通常观察不到的边界情况。

由于关于特定服务的知识主要集中在单个团队中,因此很难理解整个应用程序如何工作。

应用程序的生命周期中开发SOA平台时,可能需要所有服务更新其版本,以针对最近的平台开发。在平台发生变化后,开发人员将专注于确保应用程序功能正确,而不是引入新特性。极端的情况下,维护成本可能会急剧上升,因为那些服务不知道出现了新版本,并不断打补丁以满足平台的需求。

面向服务的体系结构遵循Conway的法则,在第2章中有介绍过。






