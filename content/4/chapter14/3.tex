
应用容器是本节的重点。虽然操作系统容器大多遵循系统编程原则,但容器带来了新的挑战和模式。此外,它们还提供专门的构建工具来应对这些挑战。我们的主要工具是Docker,它是当前构建和运行应用程序容器的标准。我们还会介绍一些构建应用程序容器的替代方法。

本节中,将重点介绍使用Docker构建和部署容器的不同方法。

\subsubsubsection{14.3.1\hspace{0.2cm}容器镜像}

描述容器镜像以及如何构建之前,理解容器和容器镜像之间的区别至关重要。这两个术语经常会有混淆,尤其是在非正式对话中。

容器和容器镜像之间的区别,就像正在运行的进程和可执行文件之间的区别一样。

容器镜像是静态的,是特定文件系统和相关元数据的快照。元数据描述在运行时设置哪些环境变量,或者在从镜像创建容器时运行哪些程序。

容器是动态的,运行容器镜像中包含的进程。可以从容器镜像创建容器,也可以通过对运行中的容器进行快照来创建容器镜像。实际上,容器镜像构建过程包括创建多个容器,在容器中执行命令,并在命令完成后对它们进行快照。

为了区分容器镜像引入的数据和运行时生成的数据,Docker使用联合挂载文件系统来创建不同的文件系统层。

这些层也出现在容器镜像中。通常,容器镜像的每个构建步骤对应于生成的容器镜像中的一个新层。

\subsubsubsection{14.3.2\hspace{0.2cm}使用Dockerfiles构建应用程序}

使用Docker构建应用程序容器镜像的最常见方法是使用Dockerfile。Dockerfile是一种命令式语言,描述生成结果图像所需的操作。一些操作创建新的文件系统层,其他则对元数据进行操作。

我们不讨论Dockerfiles相关的细节。相反,将展示载入C++应用程序的不同方法。为此,需要引入一些与Dockerfiles相关的语法和概念。

下面是一个非常简单的Dockerfile:

\begin{tcblisting}{commandshell={}}
FROM ubuntu:bionic
RUN apt-get update && apt-get -y install build-essentials gcc
CMD /usr/bin/gcc
\end{tcblisting}

通常,可以把Dockerfile分成三个部分:

\begin{itemize}
\item 
导入基础镜像(FROM指令)

\item 
在容器内执行将产生容器镜像的操作(RUN指令)

\item 
运行时使用的元数据(CMD命令)
\end{itemize}

后两部分可以相互交织,每一部分可以包含一个或多个指令。因为只有基本镜像是必需的,所以可以省略后面的部分。这并不意味着不能从一个空文件系统开始。有一个名为scratch的特殊基本镜像正是为此创建,将一个静态链接的二进制文件添加到一个空的文件系统:

\begin{tcblisting}{commandshell={}}
FROM scratch
COPY customer /bin/customer
CMD /bin/customer
\end{tcblisting}

第一个Dockerfile中,采取如下步骤:

\begin{enumerate}
\item 
导入Ubuntu Bionic基础镜像。

\item
在容器内运行命令。该命令的结果将在目标镜像中创建一个新的文件系统层。这意味着apt-get安装的包将在基于此镜像的所有容器中可用。

\item
设置运行时元数据。基于此镜像创建容器时,希望运行GCC作为默认进程。
\end{enumerate}

要从Dockerfile构建镜像,将使用\texttt{docker build}命令。它有一个必需的参数,即包含构建上下文的目录,Dockerfile本身以及想要复制到容器中的其他文件。要从当前目录构建Dockerfile,请使用\texttt{docker build}|。

这将构建一个匿名镜像,大多数时候,都希望使用已命名的镜像。在命名容器镜像时需要遵循一个约定,将在下一节中进行讨论。

\subsubsubsection{14.3.3\hspace{0.2cm}命名和分发镜像}

Docker中的每个容器镜像都有一个独特的名称,由三个元素组成:注册表名称、图像名称、标签。容器注册表是保存容器镜像的对象存储库,Docker的默认注册表是Docker.io。当从这个注册表中提取镜像时,可以省略注册表名称。

前面ubuntu:bionic的例子的全名是docker.io/ubuntu:bionic。这个例子中,ubuntu是镜像的名字,而bionic是代表镜像特定版本的标签。

基于容器构建应用程序时,可能会对存储所有注册表镜像感兴趣。可以托管私有注册中心,并将镜像保存在那里,或者使用托管解决方案。流行的托管解决方案包括:

\begin{enumerate}
\item 
Docker Hub

\item 
quay.io

\item 
GitHub

\item 
云提供商(比如AWS、GCP或Azure)
\end{enumerate}

Docker Hub仍然是最受欢迎的,不过一些公共镜像正在迁移到quay.io。两者都是通用的,并允许存储公共和私有镜像。如果已经在使用一个平台,并希望让自己的镜像接近CI流水或部署目标,那么GitHub或云提供商的服务则最有吸引力。如果想减少使用的单个服务的数量,这也会很有帮助。

如果这些解决方案都不合适,托管本地注册表也非常简单,并且需要运行单个容器。

要构建命名镜像,需要将\texttt{-t}参数传递给docker构建命令。例如,要构建一个名为\texttt{dominicanfair/merchant:v2.0.3}的镜像,可以使用\texttt{docker build -t dominicanfair/merchant:v2.0.3 ..}命令进行构建。

\subsubsubsection{14.3.4\hspace{0.2cm}已编译的应用和容器}

当使用解释语言(如Python或JavaScript)构建应用程序的容器映像时,方法基本相同:

\begin{enumerate}
\item 
安装依赖。

\item 
将源文件复制到容器镜像内。

\item 
复制必要的配置。

\item 
设置运行时命令。
\end{enumerate}

但是,对于编译的应用程序,还有一个额外的步骤,即首先编译应用程序。实现这一步骤有几种可能的方法,每种方法都有其优缺点。

最明显的方法是首先安装所有依赖项,复制源文件,然后将应用程序编译为容器构建步骤之一。主要的好处是可以精确地控制工具链的内容和配置,因此有一种可移植的方式来构建应用程序。但是,缺点太大:生成的容器镜像包含许多不必要的文件。毕竟,在运行时,既不需要源代码,也不需要工具链。由于覆盖文件系统的工作方式,不可能在前一层引入文件后删除文件。更重要的是,如果攻击者设法闯入容器,容器中的源代码可能是一个安全风险。

可以这样进行构建:

\begin{tcblisting}{commandshell={}}
FROM ubuntu:bionic
RUN apt-get update && apt-get -y install build-essentials gcc cmake
ADD . /usr/src
WORKDIR /usr/src
RUN mkdir build && \
    cd build && \
    cmake .. -DCMAKE_BUILD_TYPE=Release && \
    cmake --build . && \
    cmake --install .
CMD /usr/local/bin/customer
\end{tcblisting}

另一种方法,是在主机上构建应用程序,并且只在容器映像中复制生成的二进制文件。当一个构建过程已经建立时,这需要对当前构建过程进行更少的更改。主要缺点是必须在构建机器上匹配与容器中相同的库集。例如,如果运行的是Ubuntu 20.04作为主机操作系统,容器也必须基于Ubuntu 20.04。否则,将面临不兼容的风险。使用这种方法,还需要独立于容器配置工具链。

就像这样:

\begin{tcblisting}{commandshell={}}
FROM scratch
COPY customer /bin/customer
CMD /bin/customer
\end{tcblisting}

一种稍微复杂一点的方法是采用多阶段构建。对于多阶段构建,一个阶段可能专门用于设置工具链并编译项目,而另一个阶段将生成的二进制文件复制到目标容器映像。这比以前的解决方案有几个优点。首先,Dockerfile现在同时控制了工具链和运行时环境,因此构建的每一步都有完整的文档记录。其次,可以将镜像与工具链一起使用,以确保开发与持续集成/持续部署(CI/CD)流水之间的兼容性。这种方式也使得向工具链本身分发升级和修复更容易。主要的缺点是,容器化工具链使用起来可能不像本地工具链那样舒适。另外,构建工具并不是特别适合应用程序容器,因为应用程序容器要求每个容器都有一个进程在运行。当某些进程崩溃或强制停止时,这可能会导致意外行为。

前面例子的多阶段版本是这样的:

\begin{tcblisting}{commandshell={}}
FROM ubuntu:bionic AS builder
RUN apt-get update && apt-get -y install build-essentials gcc cmake
ADD . /usr/src
WORKDIR /usr/src
RUN mkdir build && \
    cd build && \
    cmake .. -DCMAKE_BUILD_TYPE=Release && \
    cmake --build .
FROM ubuntu:bionic
COPY --from=builder /usr/src/build/bin/customer /bin/customer
CMD /bin/customer
\end{tcblisting}

第一阶段,从第一个FROM命令开始,设置构建器,添加源文件,并构建二进制文件。然后,第二个阶段(从第二个FROM命令开始)从上一个阶段复制生成的二进制文件,而不复制工具链或源代码。

\subsubsubsection{14.3.5\hspace{0.2cm}使用清单针对多架构}

带有Docker的应用程序容器通常用于x86\_64(也称为AMD64)机器。如果只针对这个平台,那没什么好担心的。然而,如果正在开发物联网、嵌入式或边缘应用程序,可能会对多架构镜像感兴趣。

由于Docker可以在许多不同的CPU架构上使用,所以有几种方法可以在多个平台上进行镜像管理。

处理针对不同目标构建的镜像的一种方法是,使用镜像标签来描述特定的平台。可以使用\texttt{merchant:v2.0.3-aarch64},而不是\texttt{merchant:v2.0.3}。尽管这种方法看起来是最容易实现的,但实际上有点问题。

不仅必须更改构建过程,以便在标记过程中包含架构。当拉出镜像来运行时,必须手动将预期的后缀添加到所有地方。如果使用编配器,则无法以直接的方式在不同平台之间共享清单。

一种不需要修改部署步骤的方法是使用清单工具(\url{https://github.com/estesp/manifest-tool}),构建过程最初看起来与前面建议的类似。镜像在所有受支持的体系结构上分别构建,并将其推送到注册中心,其标签中带有平台后缀。在推送所有镜像之后,manifest-tool将这些映像合并为一个单一的多架构镜像。这样,每个受支持的平台都能够使用完全相同的标记。

清单工具的配置示例如下:

\begin{tcblisting}{commandshell={}}
image: hosacpp/merchant:v2.0.3
manifests:
  - image: hosacpp/merchant:v2.0.3-amd64
    platform:
      architecture: amd64
      os: linux
  - image: hosacpp/merchant:v2.0.3-arm32
    platform:
      architecture: arm
      os: linux
  - image: hosacpp/merchant:v2.0.3-arm64
    platform:
      architecture: arm64
      os: linux
\end{tcblisting}

这里,有三个受支持的平台,每个平台都有各自的后缀(\texttt{hosacpp/merchant:v2.0.3-amd64}、\texttt{hosacpp/merchant:v2.0.3-arm32}和\texttt{hosacpp/merchant:v2.0.3-arm64})。Manifest-tool结合了为每个平台构建的镜像,并生成了一个\texttt{hosacpp/merchant:v2.0.3}镜像,可以在任何地方使用。

另一种可能是使用Docker的内置功能Buildx。使用Buildx,可以附加几个构建器实例,每个实例针对所需的架构。有趣的是,不需要本地机器来运行构建,还可以在多阶段构建中使用QEMU模拟或交叉编译。尽管Buildx比之前的方法强大得多,但它也相当复杂。在编写本文时,它需要Docker实验模式和Linux内核4.8或更高版本。它需要设置和管理构建器,并不是所有东西都以直观的方式运行。有可能在不久的将来会有所改善,变得更加稳定。

准备构建环境并构建多平台镜像的示例代码如下所示:

\begin{tcblisting}{commandshell={}}
# create two build contexts running on different machines
docker context create \
    --docker host=ssh://docker-user@host1.domifair.org \
    --description="Remote engine amd64" \
    node-amd64
docker context create \
    --docker host=ssh://docker-user@host2.domifair.org \
    --description="Remote engine arm64" \
    node-arm64

# use the contexts
docker buildx create --use --name mybuild node-amd64
docker buildx create --append --name mybuild node-arm64

# build an image
docker buildx build --platform linux/amd64,linux/arm64 .
\end{tcblisting}

如果习惯于常规的docker构建命令,可能会有些困惑。

\subsubsubsection{14.3.6\hspace{0.2cm}构建应用容器的替代方法}

使用Docker构建容器映像需要Docker守护进程运行。Docker守护进程需要根权限,这可能会在某些设置中造成安全问题。即使执行构建的Docker客户端可能由非特权用户运行,但在构建环境中安装Docker守护进程并不总是可行的。

\hspace*{\fill} \\ %插入空行
\noindent
\textbf{Buildah}

Buildah是构建容器镜像的替代工具,可以配置为在没有根访问权限的情况下运行。Buildah可以与常规的dockerfile一起工作。它还提供了自己的命令行界面,可以在shell脚本或其他更直观的自动化操作中使用该界面。使用buildah接口将之前的一个dockerfile重写为shell脚本,如下所示:

\begin{tcblisting}{commandshell={}}
#!/bin/sh
ctr=$(buildah from ubuntu:bionic)
buildah run $ctr -- /bin/sh -c 'apt-get update && \
  apt-get install -y buildessential gcc'
buildah config --cmd '/usr/bin/gcc' "$ctr"
buildah commit "$ctr" hosacpp-gcc
buildah rm "$ctr"
\end{tcblisting}

Buildah的一个有趣特性是,它允许将容器镜像文件系统挂载到主机文件系统中。这样,就可以使用主机的命令与镜像的内容进行交互。如果有不希望(或由于许可限制不能)放在容器中的软件,在使用Buildah时仍然可以在容器外部调用。

\hspace*{\fill} \\ %插入空行
\noindent
\textbf{Ansible-bender}

Ansible-bender使用Ansible playbook和Buildah来构建容器镜像。所有的配置,包括基本镜像和元数据,都作为playbook中的一个变量传递。下面是之前转换成Ansible语法的例子:

\begin{tcblisting}{commandshell={}}
---
- name: Container image with ansible-bender
  hosts: all
  vars:
    ansible_bender:
      base_image: python:3-buster
    
      target_image:
        name: hosacpp-gcc
        cmd: /usr/bin/gcc
tasks:
- name: Install Apt packages
  apt:
    pkg:
      - build-essential
      - gcc
\end{tcblisting}

ansible\_bender变量负责所有特定于容器的配置,下面的任务是在基于base\_image的容器内执行的。

Ansible需要在基础镜像中提供一个Python解释器,这就是必须把之前例子中的ubuntu:bionic改为python:3-buster的原因。bionic是一个没有预装Python解释器的ubuntu镜像。

\hspace*{\fill} \\ %插入空行
\noindent
\textbf{其他}

还有其他方法可以构建容器镜像。例如,可以使用Nix创建一个文件系统镜像,然后使用Dockerfile的COPY指令将其放入镜像中。更进一步,使用其他方法准备文件系统镜像,然后使用\texttt{docker import}将其作为基本容器镜像导入。

选择适合特定需求的解决方案。请记住,使用\texttt{docker build}构建Dockerfile是最流行的方法,因此它是文档记录最好、支持最好的方法。使用Buildah更灵活,允许更好地将创建容器映像融入到构建过程中。最后,如果已经在Ansible上投入了大量成本,并且想重用已经在用的模块,ansiblebender可能是一个很好的解决方案。

\subsubsubsection{14.3.7\hspace{0.2cm}使用CMake集成容器}

本节中,将演示如何使用CMake创建Docker镜像。

\hspace*{\fill} \\ %插入空行
\noindent
\textbf{使用CMake配置Dockerfile}

首先,需要一个Dockerfile。使用另一个CMake输入文件:

\begin{lstlisting}[style=styleCMake]
configure_file(${CMAKE_CURRENT_SOURCE_DIR}/Dockerfile.in
			   ${PROJECT_BINARY_DIR}/Dockerfile @ONLY)
\end{lstlisting}

注意,如果项目是一个更大项目的一部分,则使用\texttt{PROJECT\_BINARY\_DIR}来不会覆盖源树中其他项目创建的dockerfile。

Dockerfile.in文件如下所示:

\begin{tcblisting}{commandshell={}}
FROM ubuntu:latest
ADD Customer-@PROJECT_VERSION@-Linux.deb .
RUN apt-get update && \
    apt-get -y --no-install-recommends install ./Customer-
@PROJECT_VERSION@-Linux.deb && \
    apt-get autoremove -y && \
    apt-get clean && \
    rm -r /var/lib/apt/lists/* Customer-@PROJECT_VERSION@-Linux.deb
ENTRYPOINT ["/usr/bin/customer"]
EXPOSE 8080
\end{tcblisting}

首先,指定使用最新的Ubuntu映像,在上面安装DEB包及其依赖项,然后进行整理。更新包管理器缓存的步骤和安装包的步骤是一样的,这很重要,以避免由于Docker中的层的工作方式而导致缓存过期的问题。清理也作为同一RUN命令的一部分(在同一层中)执行,以便层体积更小。安装包之后,让镜像在启动时运行客户微服务。最后,告诉Docker公开它将要监听的端口。

现在,回到CMakeLists.txt文件。

\hspace*{\fill} \\ %插入空行
\noindent
\textbf{使用CMake集成容器}

对于基于CMake的项目,可以包含负责构建容器的构建步骤。为此,需要告诉CMake找到Docker可执行文件,如果没有就退出。可以这样做:

\begin{lstlisting}[style=styleCMake]
find_program(Docker_EXECUTABLE docker)
	if(NOT Docker_EXECUTABLE)
		message(FATAL_ERROR "Docker not found")
	endif()
\end{lstlisting}

让我们重温第7章中的一个例子,为客户应用程序构建了一个二进制文件和一个Conan包。现在,希望将该应用程序打包为Debian存档,并使用客户应用程序的预安装包构建一个Debian容器镜像。

为了创建DEB包,需要一个帮助器目标。可以使用CMake的\texttt{add\_custom\_target}功能:

\begin{lstlisting}[style=styleCMake]
add_custom_target(
	customer-deb
	COMMENT "Creating Customer DEB package"
	COMMAND ${CMAKE_CPACK_COMMAND} -G DEB
	WORKING_DIRECTORY ${PROJECT_BINARY_DIR}
	VERBATIM)
add_dependencies(customer-deb libcustomer)
\end{lstlisting}

目标调用CPack来创建一个包,而忽略了其余的包。为了方便起见,希望将包创建在与Dockerfile相同的目录下。建议使用\texttt{VERBATIM}关键字,有了它CMake将会转义有问题的字符。如果没有指定,脚本的行为可能会在不同的平台上有所不同。

\texttt{add\_dependencies}调用将确保在CMake构建\texttt{customer-deb}目标之前,\texttt{libcustomer}已经构建好。现在我们有了帮助器目标,可以在创建容器镜像时使用:

\begin{lstlisting}[style=styleCMake]
add_custom_target(
	docker
	COMMENT "Preparing Docker image"
	COMMAND ${Docker_EXECUTABLE} build ${PROJECT_BINARY_DIR}
	-t dominicanfair/customer:${PROJECT_VERSION} -t
	dominicanfair/customer:latest
	VERBATIM)
add_dependencies(docker customer-deb)
\end{lstlisting}

调用Docker可执行文件来创建一个镜像,之前在包含Dockerfile和DEB包的目录中找到了Docker可执行文件。还告诉Docker将镜像标记为dominicanfair/customer:latest。最后,确定在调用Docker目标时构建DEB包。

如果使用make作为生成器,构建镜像就和\texttt{make docker}一样简单。如果喜欢使用完整的CMake命令(例如,创建与生成器无关的脚本),可以用\texttt{cmake -\,-build . -\,-target docker}。













