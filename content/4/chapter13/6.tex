
微服务的好处是,比单应用更扩展起来更容易。考虑到相同的硬件基础设施,理论上可以从微服务中获得比单体服务更多的性能。

实践中,好处的获得可没那么简单。微服务和相关的助手还提供了开销,对于较小规模的应用程序来说,性能可能不如最优的单应用。

记住,即使某件事“纸面上”看起来很好,这并不意味着它会飞起来。如果希望基于可扩展性或性能来制定架构决策,那么最好准备计算和实验。这样,相应的行为就会基于数据,而不仅仅是情绪。

\subsubsubsection{13.6.1\hspace{0.2cm}每台主机部署单个服务}

对于每个主机部署一个服务,扩展微服务需要添加或删除托管该微服务的其他机器。如果应用程序运行在云架构(公共的或私有的)上,许多提供商提供了一种称为自动扩展组的概念。

自动扩展组定义了一个基本虚拟机镜像,在所有分组实例上运行。每当达到临界阈值时(例如,CPU使用率为80\%),会创建一个新实例并将其添加到组中。由于自动扩展组在负载均衡器后面运行,因此增加的流量会在现有和新实例之间分配,从而降低每个实例的平均负载。当流量峰值减弱时,扩展控制器会关闭多余的机器,以保持低成本运行。

不同的指标可以作为扩展事件的触发器。CPU负载是最容易使用的负载之一,但它可能不是最准确的负载。其他指标(如队列中的消息数量)可能更适合相应的应用程序。

以下是扩展策略的Terraform配置的一个摘录:

\begin{lstlisting}[style=styleCXX]
autoscaling_policy {
	max_replicas = 5
	min_replicas = 3
	cooldown_period = 60
	cpu_utilization {
		target = 0.8
	}
}
\end{lstlisting}

在任何给定的时间,至少有三个实例在运行,最多有五个实例。当所有组实例的平均CPU负载达到至少80\%时,扩展器将触发。当这种情况发生时,将启动一个新实例。只有在新机器运行至少60秒(冷却期)后,才会收集它的指标。

\subsubsubsection{13.6.2\hspace{0.2cm}每台主机部署多个服务}

这种扩展模式也适用于每台主机部署多个服务的情况。但这不是最有效的方法,仅根据单个服务的吞吐量降低来扩展整个服务集类似于扩展整体。

如果正在使用这种模式,那么扩展微服务的更好方法是使用协调器。如果不想使用容器,Nomad是一个很好的选择,它可以与许多不同的执行驱动程序一起工作。对于集装箱化的工作负载,Docker Swarm或Kubernetes都是不错的助手。协调器是接下来的两章中的主题。





