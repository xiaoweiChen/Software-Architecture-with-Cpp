状态是一种设计模式,用于在对象的内部状态发生变化时改变对象的行为。不同状态的行为应该相互独立,这样添加新状态就不会影响当前状态。在有状态对象中实现所有行为的简单方法无法扩展,也不允许扩展。使用状态模式,可以通过引入新的状态类,并定义它们之间的转换来添加新的行为。本节中,将展示一种实现状态的方法,以及使用\texttt{std::variant}和静态多态双重分发模式。换句话说,将以C++的方式连接状态模式和访问者模式,从而构建一个有限的状态机。

首先,定义状态。示例中,为商店中的产品进行状态建模。如下所示:

\begin{lstlisting}[style=styleCXX]
namespace state {
	
	struct Depleted {};
	
	struct Available {
		int count;
	};

	struct Discontinued {};
} // namespace state
\end{lstlisting}

状态可以拥有自己的属性,例如:剩余物品的数量。而且,与动态多态不同,不需要从公共基类继承。可以存储在一个变量中,如下所示:

\begin{lstlisting}[style=styleCXX]
using State = std::variant<state::Depleted, state::Available,
state::Discontinued>;
\end{lstlisting}

除了状态之外,还需要事件来进行状态转换:

\begin{lstlisting}[style=styleCXX]
namespace event {
	
	struct DeliveryArrived {
		int count;
	};

	struct Purchased {
		int count;
	};

	struct Discontinued {};
	
} // namespace event
\end{lstlisting}

事件也可以具有属性,而不是继承自公共基类。现在,需要实现状态之间的转换。可以这样做:

\begin{lstlisting}[style=styleCXX]
State on_event(state::Available available, event::DeliveryArrived
delivered) {
	available.count += delivered.count;
	return available;
}

State on_event(state::Available available, event::Purchased purchased) {
	available.count -= purchased.count;
	if (available.count > 0)
	return available;
	return state::Depleted{};
}
\end{lstlisting}

如果进行了“购买”,状态可以改变,但也可以保持不变。可以使用模板一次处理多个状态:

\begin{lstlisting}[style=styleCXX]
template <typename S> State on_event(S, event::Discontinued) {
	return state::Discontinued{};
}
\end{lstlisting}

如果商品停产了,处于什么状态并不重要。现在,来实现最后一个转换:

\begin{lstlisting}[style=styleCXX]
State on_event(state::Depleted depleted, event::DeliveryArrived delivered)
{
	return state::Available{delivered.count};
}
\end{lstlisting}

需要解决的下一个难题是如何在对象中定义多个调用操作符,以便调用最佳匹配的重载。稍后使用它来调用刚刚定义的转换。助手类可以这样写:

\begin{lstlisting}[style=styleCXX]
template<class... Ts> struct overload : Ts... { using Ts::operator()...; };
template<class... Ts> overload(Ts...) -> overload<Ts...>;
\end{lstlisting}

创建重载结构,该结构将使用变量模板、折叠表达式和类模板参数推导指南提供在构造过程中,传递给所有调用操作符。要了解更深入的解释,以及实现访问的另一种方法,请参阅Bartłomiej Filipek在扩展阅读部分的博客文章。

现在可以开始实现状态机本身了:

\begin{lstlisting}[style=styleCXX]
class ItemStateMachine {
public:
	template <typename Event> void process_event(Event &&event) {
		state_ = std::visit(overload{
			[&](const auto &state) requires std::is_same_v<
			decltype(on_event(state, std::forward<Event>(event))), State> {
				return on_event(state, std::forward<Event>(event));
			},
			[](const auto &unsupported_state) -> State {
				throw std::logic_error{"Unsupported state transition"};
			}
		},
		state_);
	}

private:
	State state_;
};
\end{lstlisting}

\texttt{process\_event}函数将接受任何事件,将使用当前状态和传递的事件调用适当的\texttt{on\_event}函数,并切换到新的状态。如果给定的状态和事件可以找到\texttt{on\_event}重载,将调用第一个lambda。否则,约束将不满足,第二个更通用的重载将调用。这意味着如果存在不受支持的状态转换,则会抛出异常。

现在,提供一种方法来报告当前状态:

\begin{lstlisting}[style=styleCXX]
std::string report_current_state() {
	return std::visit(
	overload{[](const state::Available &state) -> std::string {
			return std::to_string(state.count) +
			" items available";
		},
		[](const state::Depleted) -> std::string {
			return "Item is temporarily out of stock";
		},
		[](const state::Discontinued) -> std::string {
			return "Item has been discontinued";
	}},
	state_);
}
\end{lstlisting}

这里,使用重载来传递三个lambda,每个lambda都返回一个通过访问状态对象生成的字符串报告。

应用解决方案:

\begin{lstlisting}[style=styleCXX]
auto fsm = ItemStateMachine{};
std::cout << fsm.report_current_state() << '\n';
fsm.process_event(event::DeliveryArrived{3});
std::cout << fsm.report_current_state() << '\n';
fsm.process_event(event::Purchased{2});
std::cout << fsm.report_current_state() << '\n';
fsm.process_event(event::DeliveryArrived{2});
std::cout << fsm.report_current_state() << '\n';
fsm.process_event(event::Purchased{3});
std::cout << fsm.report_current_state() << '\n';
fsm.process_event(event::Discontinued{});
std::cout << fsm.report_current_state() << '\n';
// fsm.process_event(event::DeliveryArrived{1});
\end{lstlisting}

运行时,将产生如下输出:

\begin{tcblisting}{commandshell={}}
Item is temporarily out of stock
3 items available
1 items available
3 items available
Item is temporarily out of stock
Item has been discontinued
\end{tcblisting}

除非用不支持的转换取消最后一行的注释,否则将在最后抛出异常。

尽管支持的状态和事件列表仅限于编译时提供的状态和事件列表,但与基于动态多态的解决方案相比,解决方案性能要好得多。更多关于状态、变量和各种访问方式的信息,请参见Mateusz Pusz在2018年CppCon上的讲话,该讲话在扩展阅读部分可以找到。

结束这一章之前,最后了解一下如何处理内存。
