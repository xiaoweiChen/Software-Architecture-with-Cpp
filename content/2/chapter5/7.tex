本章中,了解了许多C++特性,以及对编写简洁、有表现力和性能的影响,还了解了如何提供正确的C++组件接口。现在,可以应用诸如RAII之类的原则来编写不受资源泄漏影响的优雅代码。还知道如何利用\texttt{std::optional}之类的类型在接口中,更好地表达意图。

接下来,演示了如何使用泛型和模板lambda等特性,以及如何使用\texttt{if constexpr}来编写可用于多种类型的(更少)代码。现在,还能够使用指定的初始化器以清晰的方式定义对象。之后,了解了如何使用标准范围以声明式风格编写简单的代码,如何使用\texttt{constexpr}编写可在编译时和运行时执行的代码,以及如何使用概念编写受约束的模板代码。

最后,演示了如何用C++模块编写模块化代码。下一章中,将讨论如何设计C++代码,以便在可用的习惯用法和模式的基础上进行构建。