
C++严重依赖于编写类型安全代码的机制。语言构造(如显式构造函数和转换操作符)已经融入语言中很长时间了。越来越多的安全类型被引入标准库。有\texttt{optional},以帮助避免引用空值,\texttt{string\_view},以帮助您避免超出范围,和作为类型的安全包装,这里仅举几个例子。此外,由于其零成本抽象,建议创建自己的有用且不可能滥用的类型。

通常,使用C风格的结构会导致类型不安全。一个例子就是C的强制转换,可以对应为\texttt{const\_cast}、\texttt{static\_cast}、\texttt{reinterpret\_cast},或者两者之一与\texttt{const\_cast}结合使用。若不小心对\texttt{const}对象使用\texttt{const\_cast},则会出现未定义行为。从\texttt{reinterpret\_cast<T>}中读取返回的内存地址也是会如此,如果\texttt{T}不是对象的原始类型(C++20添加了\texttt{std::bit\_cast})。如果使用C++的类型转换,这两种情况都更容易避免。

C在类型方面太宽容了。幸运的是,C++为有问题的C结构体引入了许多类型安全的替代方案。用\texttt{streams}和\texttt{std::format}来代替\texttt{printf}等,还有\texttt{std::copy}和其他类似的算法来代替不安全的\texttt{memcpy}。最后,用模板代替函数使用\texttt{void *}(并在性能方面付出代价)。在C++中,模板通过一个称为概念的特性获得了更多的安全性。来看看如何通过使用它来改进代码。

\subsubsubsection{5.5.1\hspace{0.2cm}约束模板参数}

概念可以改进代码的第一种方式是使其更通用。还记得需要在一个地方更改容器类型,从而导致在其他地方也发生了更改的情况吗?如果没有将容器更改为具有完全不同语义的容器,并且必须以不同的方式使用,这意味着代码可能不够通用。

另外,是否曾经编写过模板或在代码中加入\texttt{auto},然后想知道如果改变了底层类型,代码是否会崩溃?

概念是关于在操作类型上放置正确级别的约束。它们限制模板可以匹配的类型,并在编译时进行检查。例如,假设写了以下内容:

\begin{lstlisting}[style=styleCXX]
template<typename T>
void foo(T& t) {...}
\end{lstlisting}

现在,可以这样写:

\begin{lstlisting}[style=styleCXX]
void foo(std::swappable auto& t) {...}
\end{lstlisting}

这里,\texttt{foo()}必须传递一个支持\texttt{std::swap}的类型才能工作。还记得一些匹配太多类型的模板吗?以前,可以使用\texttt{std::enable\_if}、\texttt{std::void\_t}或\texttt{if constexpr}来约束它们。然而,编写\texttt{enable\_if}语句有些麻烦,可能会减慢编译时间。这里,由于概念的简洁性以及可以清晰地表达其意图,概念再次发挥了作用。

C++20中有几十个标准概念,大多数存在于头文件\texttt{<concepts>}中,可以分为四类:

\begin{itemize}
\item 
核心语言概念,如\texttt{derived\_from}、\texttt{integral}、\texttt{swappable}和\texttt{move\_constructible}

\item 
比较的概念,比如布尔可测试、\texttt{equality\_comparable\_with}和\texttt{totally\_ordered}

\item 
对象概念,如可移动、可复制、半正则和正则

\item 
可调用的概念,如可调用、谓词和\texttt{strict\_weak\_order}
\end{itemize}

其他的定义在\texttt{<iterator>}头文件中。这些可以分为以下几类:

\begin{itemize}
\item 
间接可调用的概念,例如\texttt{indirect\_binary\_predicate}和\texttt{indirectly\_unary\_invocable}

\item 
常见的算法要求,例如\texttt{indirectly\_swappable}, \texttt{permutable}, \texttt{mergeable}和\texttt{sortable}
\end{itemize}

最后,可以在\texttt{<ranges>}头文件中找到一堆相关特性。包括\texttt{range(duh)}、\texttt{continuous\_range}和\texttt{view}。

如果这还不能满足需要,可以声明自己的概念,类似于标准定义中提到的方式。例如,\texttt{movable}的概念是这样实现的:

\begin{lstlisting}[style=styleCXX]
template <class T>
concept movable = std::is_object_v<T> && std::move_constructible<T> &&
std::assignable_from<T&, T> && std::swappable<T>;
\end{lstlisting}

此外,如果查看\texttt{std::swappable}的实现,会看到以下内容:

\begin{lstlisting}[style=styleCXX]
template<class T>
concept swappable = requires(T& a, T& b) { ranges::swap(a, b); };
\end{lstlisting}

如果\texttt{range::swap(a, b)}编译了该类型的两个引用,类型\texttt{T}将是可交换的。

\begin{tcolorbox}[colback=webgreen!5!white,colframe=webgreen!75!black, title=TIP]
\hspace*{0.7cm}定义自己的概念时,请确保涵盖了语义需求。定义接口时指定和使用一个概念是对该接口的使用者作出的承诺。
\end{tcolorbox}

为了简洁,可以在声明中使用所谓的简写符号:

\begin{lstlisting}[style=styleCXX]
void sink(std::movable auto& resource);
\end{lstlisting}

为了可读性和类型安全,建议将auto与概念一起使用,以约束类型,并让使用者知道他们正在处理的对象的类型,以这种方式编写的代码将保留自动式通用性的优点。也可以在常规函数和lambda中使用它。

使用概念的好处是更短的错误信息。将关于一个编译错误的几十行代码,可以缩减为几行代码是很常见的。另一个好处是,可以尽可能多地使用概念。

现在,让我们回到多米尼加博览会的例子。这一次,添加一些概念,看它们如何改进原来的实现。

首先,让\texttt{get\_all\_featured\_items}返回一个商品range。这里,可以将概念添加到返回类型来实现:

\begin{lstlisting}[style=styleCXX]
range auto get_all_featured_items(const Stores &stores);
\end{lstlisting}

到目前为止,一切顺利。现在,让向这个类型添加另一个需求,当调用\texttt{order\_items\_by\_date\_added}时将强制执行,所以range必须是可排序的。

\texttt{std::sortable}已经为range迭代器定义过了,但是为了方便起见,定义一个新的概念\texttt{sortable\_range}:

\begin{lstlisting}[style=styleCXX]
template <typename Range, typename Comp, typename Proj>
concept sortable_range =
	random_access_range<Range> &&std::sortable<iterator_t<Range>, Comp,
Proj>;
\end{lstlisting}

与标准库类似,可以接受比较器和投影(通过range引入)。满足\texttt{random\_access\_range}概念的类型要求(将由满足\texttt{random\_access\_range}概念的类型匹配),以及满足上述可排序概念的迭代器。就是这么简单!

定义概念时,还可以使用\texttt{requires}子句指定其他约束。例如,如果想要range只存储一个\texttt{date\_added}成员的元素,可以这样写:

\begin{lstlisting}[style=styleCXX]
template <typename Range, typename Comp>
concept sortable_indirectly_dated_range =
	random_access_range<Range> &&std::sortable<iterator_t<Range>, Comp> &&
requires(range_value_t<Range> v) { { v->date_added }; };
\end{lstlisting}

在例子中,不需要对类型进行那么多的约束,因为在使用概念和定义时,应该保留一定的灵活性,以便重用。

这里重要的是,可以使用\texttt{requires}子句指定当类型满足某个概念的要求时,应该调用哪些代码。如果愿意,可以为每个子表达式返回的类型指定约束。例如,要定义一个可递增的数时,可以使用以下语句:

\begin{lstlisting}[style=styleCXX]
requires(I i) {
	{ i++ } -> std::same_as<I>;
}
\end{lstlisting}

现在有了自己的概念,可以用来重新定义\texttt{order\_items\_by\_date\_added}函数:

\begin{lstlisting}[style=styleCXX]
void order_items_by_date_added(
sortable_range<greater, decltype(&Item::date_added)> auto &items) {
	sort(items, greater{}, &Item::date_added);
}
\end{lstlisting}

现在,编译器将检查传递给它的range是否可排序,并包含一个可以使用\texttt{std::ranges::greater\{\}}进行排序的\texttt{date\_added}成员。

若在这里使用更受约束的概念,函数看起来会像这样:

\begin{lstlisting}[style=styleCXX]
void order_items_by_date_added(
sortable_indirectly_dated_range<greater> auto &items) {
	sort(items, greater{}, &Item::date_added);
}
\end{lstlisting}

最后,来重写呈现函数:

\begin{lstlisting}[style=styleCXX]
template <input_range Container>
requires std::is_same_v<typename Container::value_type,
		gsl::not_null<const Item *>> void
render_item_gallery(const Container &items) {
	copy(items,
	std::ostream_iterator<typename Container::value_type>(std::cout,
	"\n"));
}
\end{lstlisting}

这里,可以看到在模板声明中可以使用概念名来代替\texttt{typename}关键字。再下一行,可以看到\texttt{requires}关键字还可以用于根据特征进一步约束类型。如果不想指定一个新概念,也很方便。

这就是概念。现在,接下来编写一些模块化的C++代码。

