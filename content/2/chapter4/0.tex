
Patterns help us deal with complexity. At the level of a single software component, you can use software patterns such as the ones described by the four authors of the book (better known as the \textit{Gang of Four}) \textit{Design Patterns: Elements of Reusable Object-Oriented Software}. When we move higher up and start looking at the architecture between different components, knowing when and how to apply architectural patterns can go a long way.

There are countless such patterns that are useful for different scenarios. In fact, to even get to know all of them, you would need to read more than just one book. That being said, we selected several patterns for this book, suited for achieving various architectural goals.

In this chapter, we'll introduce you to a few concepts and fallacies related to architectural design; we'll show when to use the aforementioned patterns and how to design highquality components that are easy to deploy.


本章将讨论以下内容:

\begin{itemize}
\item The different service models and when to use each of them
\item How to avoid the fallacies of distributed computing
\item The outcomes of the CAP theorem and how to achieve eventual consistency
\item Making your system fault-tolerant and available
\item Integrating your system
\item Achieving performance at scale
\item Deploying your system
\item Managing your APIs

\end{itemize}

By the end of this chapter, you'll know how to design your architecture to provide several important qualities, such as fault tolerance, scalability, and deployability. Before that, let's first learn about two inherent aspects of distributed architectures.



























