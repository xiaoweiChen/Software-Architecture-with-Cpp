
适当的API对于开发团队和产品的成功至关重要。可以将这个主题分成两个小主题:系统级API和组件级API。本节中,将讨论系统级别上处理API,而下一章将提供关于组件级别的技巧。

除了管理对象之外,还需要管理整个API。如果想要引入有关API使用的策略,控制对所述API的访问,收集性能指标和其他分析数据,或者只是根据客户使用的接口向他们收费,\textbf{API管理(APIM)}就是问题的解决方案。

通常一组APIM工具包含以下组件:

\begin{itemize}
\item 
\textbf{API网关}: API的所有用户的单一入口。下一节将对此进行更多介绍。

\item 
\textbf{报告和分析}: 监控API、资源消耗或数据发送的性能和延迟。可以利用这些工具来检测使用趋势,了解API的哪些部分和背后的哪些组件是性能瓶颈,或者可以提供哪些SLA,以及如何改进。

\item 
\textbf{开发者门户}: 帮助用户快速了解API,并使用API。

\item 
\textbf{管理员门户}: 管理策略、用户,并将API打包成可销售的产品。

\item  
\textbf{货币化}: 根据客户使用API的方式向他们收费,并帮助相关的业务流程。
\end{itemize}

APIM工具由云提供商和独立方提供,例如NGINX的Controller或Tyk。

在为给定的云设计API时,云提供商通常会提供的良好实践。例如,可以在\textit{扩展阅读}部分找到Google云平台的通用设计模式,其许多实践都围绕着Protobuf进行。

选择正确的使用API方式可以走的更远,向服务器提交请求的最简单方法是直接连接到服务。虽然对于小型应用程序来说很容易设置,但可能会导致性能问题。API使用者可能需要调用几个不同的服务,从而导致高延迟。使用这种方法也不可能实现适当的扩展性。

更好的方法是使用API网关。此类网关通常是APIM解决方案的重要组成,也可以单独使用。

\subsubsubsection{4.6.1\hspace{0.2cm}API网关}

API网关是希望使用API的客户端的入口点,可以将传入的请求转到特定的实例或服务集群。因为不再需要知道所有后端节点,或者如何相互协作,所以这可以简化客户端的代码。客户端只需要知道API网关的地址——网关将处理其余的事情。由于对客户机隐藏了后端架构,因此可以在不涉及客户机代码的情况下对其进行重构。

网关可以将系统API的多个部分聚合,然后使用\textbf{7层路由}(例如,基于URL)到系统的部分。7层路由是由两个云提供商提供的,还有Envoy等工具。

与本章中描述的许多模式一样,始终要考虑是否需要引入另一个模式来增加架构的复杂性。考虑添加它将如何影响可用性、容错和性能(如果很重要的话)。毕竟,网关通常只是单个节点,所以不要让它成为瓶颈或故障点。

前面几章提到的后端前端模式可以认为是API网关模式的一个变体。在后端用于前端的情况下,每个前端都可以连接到自己的网关。

了解了系统设计与API设计之间的关系,就来总结一下在本章中讨论的内容吧。











