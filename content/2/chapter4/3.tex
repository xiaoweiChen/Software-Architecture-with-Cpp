
可用性和容错是软件质量,至少对每个架构都有一定的重要性。如果系统无法做到,那创建软件系统还有什么意义呢?本节中,将了解这些术语的确切含义,以及在解决方案中提供的技巧。

\subsubsubsection{4.3.1\hspace{0.2cm}计算系统的可用性}

可用性是系统正常运行、可用和可访问的时间百分比。导致系统无法响应的崩溃、网络故障或极高的负载(例如,来自DDoS攻击)都会影响其可用性。

通常,尽可能高的可用性是一个不错的指标。可能会偶然发现\textit{数9}的术语,因为可用性通常指定为99\%(两个9),99.9\%(三个)等。每增加一个9就更难获得,所以在做出承诺时要小心。看看下面的表格,看看如果按月计算,能承受多长时间的停机时间:

\begin{table}[H]
	\begin{tabular}{|l|l|}
		\hline
		\multicolumn{1}{|c|}{\textbf{停机时间/月}} & \multicolumn{1}{c|}{\textbf{正常运行时间}} \\ \hline
		7 小时 18 分钟                            & 99\% (“两个9”)                   \\ \hline
		43 分钟 48 秒                         & 99.9\% (“三个9”)               \\ \hline
		4 分钟 22.8 秒                        & 99.99\% (“四个9”)               \\ \hline
		26.28 秒                                 & 99.999\% (“五个9”)              \\ \hline
		2.628 秒                                 & 99.9999\% (“六个9”)              \\ \hline
		262.8 毫秒                                      & 99.99999\% (“七个9”)           \\ \hline
		26.28 毫秒                                      & 99.999999\% (“八个9”)          \\ \hline
		2.628 毫秒                                      & 99.9999999\% (“九个9”)          \\ \hline
	\end{tabular}
\end{table}

云应用程序会提供\textbf{服务级别协议(SLA)},它指定每一给定时间段(例如一年)出现停机的时间,云服务的SLA将依赖于构建云服务的SLA。

要计算需要协作的两个服务之间的复合可用性,应该将它们的正常运行时间相乘。这意味着如果有两个可用性为99.99\%的服务,复合可用性将为99.99\% * 99.99\% = 99.98\%。要计算冗余服务(例如两个独立区域)的可用性,应该将它们的不可用性相乘。例如,如果两个区域有99.99\%可用性,它们的总不可用性将是(100\% - 99.99\%)*(100\% - 99.99\%)= 0.01\% * 0.01\% = 0.0001\%,因此复合可用性是99.9999\%。

这里,不可能提供100\%的可用性。故障确实会不时地发生,所以系统需要容错。

\subsubsubsection{4.3.2\hspace{0.2cm}构建容错系统}

容错是系统检测故障,并优雅地处理故障的能力。因为由于云的性质,有许多可能会突然出现各种各样的问题,所以基于云的服务必须具有容错性。良好的容错能力有助于提高服务的可用性。

不同类型的问题需要不同的处理方法:从预防到检测,再到最小化影响。先从避免单点故障的方法开始。

\hspace*{\fill} \\ %插入空行
\noindent
\textbf{替代装置}

最基本预防方式是引入\textbf{替代装置}。类似于为汽车准备一个备用轮胎,这样就有了一份服务的副本,当主服务器出现故障时,可以由副本接管。这种方式也称为\textbf{失效备援}。

备用服务器如何了解何时介入?实现的一种方法是使用\textit{检测故障}中的心跳机制。

为了使切换更快,可以将所有进入主服务器的消息也发送到备用服务器,这称为\textbf{热备份}。好的方式是保留最后一条消息,如果这条消息\textit{有毒},并杀死了主服务器,备份服务器可以直接拒绝它。

上述机制称为\textbf{主动}(或\textbf{主从})故障转移,因为备份服务器不处理传入的流量。如果是这样,将有一个\textbf{双控}(或\textbf{双主})故障转移。有关双控架构的更多信息,请参考\textit{扩展阅读}中的最后一个链接。

确保在发生故障转移时不会丢失数据,使用带有备份存储的消息队列可能有助于解决这一问题。

\hspace*{\fill} \\ %插入空行
\noindent
\textit{领袖选举}

对于两个服务器来说,了解两个服务器各自的任务也很重要——如果两个服务器开始都作为主实例运行,就可能会遇到麻烦。选择主服务器称为领袖选举模式。有几种方法可以做到这一点,例如:引入第三方仲裁者,通过竞争获得共享资源的独占所有权,通过选择级别最低的实例,或通过使用恶霸选举或令牌环选举等算法。

领袖选举也是下一个相关概念的重要组成部分:达成共识。

\hspace*{\fill} \\ %插入空行
\noindent
\textit{达成共识}

如果希望系统即使在网络分区发生,或某些服务实例出现故障时也能运行,需要一种方法使实例之间达成共识。必须同意承担什么样的风险,以及以相应的顺序。一种简单的方法是允许每个实例对正确的状态进行投票。然而,在某些情况下,这不足以正确地或根本无法达成一致。另一种方法是选举一个领袖,让他宣传自己的能力。因为手工实现这样的算法并不容易,所以建议使用行业验证的主流共识协议,如Paxos和Raft。后者会更受欢迎,因为它更简单,更容易理解。

现在让我们讨论另一种防止系统出错的方法。

\hspace*{\fill} \\ %插入空行
\noindent
\textbf{重复}

这种方法在数据库中尤为流行,并且有助于扩展数据库。\textbf{重复}意味着将运行服务的几个实例与重复的数据并行,其也会处理所有的传入流。

\begin{tcolorbox}[colback=blue!5!white,colframe=blue!75!black, title=Note]
\hspace*{0.75cm}不要混淆复制和分片。后者不需要冗余数据,可以在带来良好的性能。如果正在使用Postgres,那可以试试Citus(\url{https://www.citusdata.com})。
\end{tcolorbox}

对于数据库,有两种方式可以进行重复。

\hspace*{\fill} \\ %插入空行
\noindent
\textbf{单主架构}

这个场景中,所有服务器都能够执行只读操作,但是只有一个主服务器也可以写。数据从主节点通过从节点进行复制,可以采用一对多拓扑,也可以采用树状拓扑。如果主服务器发生故障,系统仍然可以以只读模式运行,直到故障解决。

\hspace*{\fill} \\ %插入空行
\noindent
\textbf{多主架构}

还可以使用一个具有多个主服务器的系统。如果有两个服务器,可以进行\textit{双主重复}的方案。如果其中一个服务器挂了,其他服务器仍然可以正常运行。但现在要么同步写操作,要么提供更宽松的一致性保证。另外,还需要一个\textbf{负载均衡器}。

这类重复的例子还包括微软的Active Directory、OpenLDAP、Apache的CouchDB或Postgres-XL。

现在了解一下两种防止过高负载(会导致故障)的方法。

\hspace*{\fill} \\ %插入空行
\noindent
\textbf{基于队列的负载均衡}

此策略旨在减少系统负载突然激增的影响。请求泛滥会导致性能问题、可靠性问题,甚至丢失有效的请求。还有,队列是对该场景进行缓和。

要实现此模式,只需要引入队列,以便异步添加传入请求。可以使用Amazon的SQS、Azure的服务总线、Apache Kafka、ZeroMQ或其他队列来实现。

现在,不是在传入请求中出现峰值,而是平均负载。服务可以从上述队列中获取请求,并在不知道负载增加的情况下处理它们,就这么简单。

如果是高性能队列,并且任务可以并行化,则此模式的另一个好处是有更好的扩展性。

此外,如果服务不可用,当服务恢复时,请求仍然会添加到该服务的队列中进行处理,因此这可能是一种有助于提高可用性的方法。

如果请求不经常出现,考虑将服务实现为仅在队列中有项目时运行的函数,以节省成本。

使用此模式时,总延迟将随着队列的增加而增加。Apache Kafka和ZeroMQ可以提供低延迟队列,但如果这是一个问题,需要另一种方法来处理负载。

\hspace*{\fill} \\ %插入空行
\noindent
\textbf{反压力}

如果负载仍然很高,很可能是有更多的任务超出了服务器处理能力。这可能会导致缓存丢失和交换异常(如果请求不再适合内存),以及删除请求和其他麻烦的事情。如果预计会有沉重的负担,反压力可能是一个更好的解决办法。

反向压力意味着,不会在每个传入请求时对服务施加更大的压力,而是将其推回调用方,以便调用方需要处理这种情况。有几种不同的方法可以做到这一点。

例如,可以阻塞接收网络数据包的线程。调用者将发现无法将请求推到服务——相反,其实是我们将压力推回到上游。

另一种方法是,识别更大的负载并简单地返回错误代码,例如:503。可以为架构建模,以便由另一个服务来完成请求。Envoy代理(Envoy Proxy,\url{https://envoyproxy.io})就是这样一个服务,许多其他场合都会用到。

Envoy代理可以应用基于预定义配额的反压力,所以服务实际上永远不会超载。它还可以测量处理请求和仅在超过某个阈值时,施加反压力所需的时间;还有许多其他情况会返回各种错误代码。希望使用者有提前的计划,从而明确如果压力返回来应该怎么办。

既然已经知道了如何防止故障,再了解一下如何在故障发生时检测故障。

\subsubsubsection{4.3.3\hspace{0.2cm}检测故障}

正确而快速的故障检测可以省去很多麻烦,而且往往还能节约成本。有许多方法可以根据不同的需要检测故障。让我们来过一下。

\hspace*{\fill} \\ %插入空行
\noindent
\textbf{边车设计模式}

Envoy代理可以说这是\textbf{边车设计模式}的一个例子。该模式在很多情况下都很有用,而不仅仅是错误预防和检测。

通常,边车允许向服务添加许多功能,而不需要编写额外的代码。类似地,物理边车可以连接到摩托车上,软件边车则可以连接到服务上——这两种情况下都可以对功能进行扩展。

边车如何帮助检测故障?首先,通过提供健康检查功能。对于被动运行状况检查,特使代理可以检测服务集群中的实例是否已经开始行为异常,这叫做\textbf{离群检测}。Envoy代理可以查找连续的5XX错误代码、网关故障等。除了检测这些错误之外,还可以弹出它们,以便整个集群保持健康。

Envoy代理还提供了主动的运行状况检查,这意味着它可以探测服务本身,而不仅仅是观察它对传入流的反应。

本章中,展示了一般的边车模式的其他用法,特别是特使代理。现在来讨论另一种故障检测机制。

\hspace*{\fill} \\ %插入空行
\noindent
\textbf{心跳机制}

最常见的故障检测方法是通过\textbf{心跳机制}。一个\textbf{心跳}是在两个服务之间定期(通常是几秒钟)发送的信号或消息。

如果几个连续的心跳没有了,接收服务可以认为发送服务\textbf{挂了}。前面几节中提到的主备份服务的情况下,这可能导致故障转移。

实现心跳机制时,要确保可靠性。误报可能会造成麻烦,因为服务可能会混淆,例如:哪个服务应该成为新的主服务。一个好方法是仅为心跳提供一个单独的端点,这样就不会那么受到常规端点流的影响。

\hspace*{\fill} \\ %插入空行
\noindent
\textbf{漏桶计数器}

另一种检测错误的方法是添加\textbf{漏桶}计数器。对于每个错误,计数器将增加,并且在达到某个阈值(桶已满)后,将发出错误信号并处理。在固定的时间间隔内,计数器将减少(因此,漏桶了)。这样,只有在短时间内出现很多错误时,才会认为是故障。

如果有时出现错误是很正常的(例如:在处理网络时),这种模式会很有用。

既然已经知道了如何检测故障,就继续了解在故障发生时应该做什么。

\subsubsubsection{4.3.4\hspace{0.2cm}降低故障的影响}

检测正在发生的故障需要花费时间,而解决故障则需要更多宝贵的资源。这就是为什么应该努力将错误的影响降到最低。

\hspace*{\fill} \\ %插入空行
\noindent
\textbf{重试调用}

当应用程序调用另一个服务时,有时调用会失败。对于这种情况,最简单的补救方法就是重试。如果错误是暂时的,而没有重试,那么该错误可能会在系统中传播,造成更大的损失。实现自动重试此类调用可以为省去很多麻烦。

还记得Envoy代理吗?它可以代表管理者执行自动重试,从而无需对源代码进行更改。

例如,在Envoy代理中可以添加一个重试策略的配置:

\begin{tcblisting}{commandshell={}}
retry_policy:
  retry_on: "5xx"
  num_retries: 3
  per_try_timeout: 2s
\end{tcblisting}

如果Envoy代理返回错误,比如:映射到5XX代码的503 HTTP代码或gRPC错误,则将使Envoy代理重试调用。将有三次重试,如果没有在2秒内完成,每次认为失败。

\hspace*{\fill} \\ %插入空行
\noindent
\textbf{避免级联故障}

如果没有重试,错误将会传播,在整个系统中造成一连串的失败。现在来了解更多防止这种情况发生的方法。

\hspace*{\fill} \\ %插入空行
\noindent
\textit{断路器}

\textbf{断路器的模式}是一个非常有用的工具。它允许快速地注意到某个服务无法处理请求,因此对它的调用可能会短路。这可以发生在调用方附近(Envoy代理提供了这样的功能),也可以发生在调用方(减少了调用时间)。在使用Envoy代理的情况下,可以添加以下配置:

\begin{tcblisting}{commandshell={}}
circuit_breakers:
  thresholds:
    - priority: DEFAULT
      max_connections: 1000
      max_requests: 1000
      max_pending_requests: 1000
\end{tcblisting}

这两种情况下,对服务的调用引起的负载可能会下降,这在某些情况下可以帮助服务恢复正常操作。

如何在调用方实现一个断路器?在进行了一些调用之后,例如:漏桶溢出了,可以在指定的一段时间内停止进行新的调用(例如,直到桶不再溢出)。

\hspace*{\fill} \\ %插入空行
\noindent
\textit{隔板}

另一种限制断层扩散的方法是直接从隔板中获取。当建造船只时,通常不希望船体上有一个洞,让船充满水。为了减少这些洞的破坏,可以把船体分成几个舱壁,每一个舱壁都很容易分开。在这种情况下,只有损坏的舱壁才会充满水。

同样的原则也适用于限制软件架构中的故障影响。可以将实例划分到组中,也可以将其使用的资源分配到组中。设置配额也可以视为该模式的一个示例。

可以为不同的用户组创建单独的舱壁,如果需要对它们进行优先级划分,或者为关键用户提供不同级别的服务,这会很有用。

\hspace*{\fill} \\ %插入空行
\noindent
\textbf{多地部署}

最后一种方法称为\textbf{多地部署},这个名字来自地理节点。当服务部署在多个地区时,可以使用它。

当一个区域发生故障时,可以将流重定向到其他不受影响的区域。当然,与调用同一数据中心中的其他节点相比,这将导致更高的延迟,但通常将不太重要的用户重定向到远程区域要比完全失败调用要好得多。

既然已经知道了如何通过系统体系结构提供可用性和容错,接下来就来了解一下如何将其组件集成在一起。





















