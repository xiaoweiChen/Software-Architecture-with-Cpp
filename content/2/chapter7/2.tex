
编译器是每个开发者工作中最重要的工具之一,这就是为什么了解他们非常重要。本节中,将介绍一些有用的技巧。这只是冰山一角,因为关于这些工具及其各种可用标志、优化、功能和其他细节汇总起来就是可以成书的量级,GCC还有有一个专门的Wiki页面,里面有关于编译器的书籍列表!可以在本章末尾的扩展阅读中找到它。

\subsubsubsection{7.2.1\hspace{0.2cm}多编译器}

构建过程中,需要考虑使用多个编译器,而非一个,原因是它们可以检测代码中的不同问题。例如,MSVC默认启用签名检查。使用多个编译器可以提前发现可能遇到的可移植性问题,特别是决定在不同的操作系统上编译代码时,例如:从Linux迁移到Windows或其他。要无痛完成这些工作,需要编写可移植的、符合ISO C++的代码。Clang是一个比较好的选择,它力求比GCC更符合C++标准。如果正在使用MSVC,可以试着添加\texttt{/permissive-}选项(在Visual Studio 17加入,对于使用15.5+版本创建的项目默认启用)。对于GCC,在为代码选择C++标准时尽量不要使用GNU扩展(例如,使用\texttt{-std=c++17}而不是\texttt{-std=gnu++17})。如果目标是性能,那么使用各种编译器来构建也可以为特定的用例,选择运行最快的二进制文件。

\begin{tcolorbox}[colback=webgreen!5!white,colframe=webgreen!75!black, title=TIP]
\hspace*{0.7cm}无论为发布版本选择哪种编译器,都可以考虑使用Clang进行开发。它可以在macOS、Linux和Windows上运行,支持与GCC相同的标志集,旨在提供最快的构建时间和简明的编译错误。
\end{tcolorbox}

如果正在使用CMake,有两种常见的方法来添加另一个编译器。一种是在调用CMake时传递相应的编译器,像这样:

\begin{tcblisting}{commandshell={}}
mkdir build-release-gcc
cd build-release-gcc
cmake .. -DCMAKE_BUILD_TYPE=Release -DCMAKE_C_COMPILER=/usr/bin/gcc \
  -DCMAKE_CXX_COMPILER=/usr/bin/g++
\end{tcblisting}

也可以在调用CMake之前设置\texttt{CC}和\texttt{CXX},但这些环境变量在不同的平台(如macOS)上有所不同。

另一种方法是使用工具链文件。如果只是需要使用不同的编译器,这可能没必要,但若想要交叉编译时,这是首选的解决方案。要使用一个工具链文件,应该作为一个参数传递给CMake:\texttt{-DCMAKE\_TOOLCHAIN\_FILE=toolchain.cmake}。

\subsubsubsection{7.2.2\hspace{0.2cm}减少构建时间}

每年,程序员都要花费无数个小时等待他们的构建完成。减少构建时间是提高整个团队生产力的一种简单方法,所以这里讨论一些实现它的方法。

\hspace*{\fill} \\ %插入空行
\noindent
\textbf{快速编译器}

提高构建速度的最简单方法之一就是升级编译器。例如,通过将Clang升级到7.0.0,可以使用预编译头(PCH)文件节省多达30\%的构建时间。从Clang 9开始,就有了了\texttt{-ftime-trace}选项,该选项可以提供有关它处理的所有文件的编译时间的信息。其他编译器也有类似的开关:查看GCC的\texttt{-ftime}报告或MSVC的/Bt和\texttt{/d2cgsummary}。通常可以通过切换编译器来获得更快的编译速度,这在开发机器上特别有用,例如:Clang编译代码的速度通常比GCC快。

当有了一个快速的编译器后,再来看看它需要编译什么。

\hspace*{\fill} \\ %插入空行
\noindent
\textbf{反思模板}

编译过程的不同部分需要不同的时间来完成,这对于编译时构造尤其重要。Odin Holmes的实习生之一Chiel Douwes在对各种模板操作的编译时成本进行基准测试的基础上创建了所谓的Chiel规则。这一点,以及其他基于类型的模板元编程技巧,可以在Odin Holmes的《基于类型的模板元编程并未消亡》讲座中看到。从最快到最慢,排列如下:

\begin{itemize}
\item 
查找已实例化的类型(例如,模板实例化)

\item 
将参数传递给别名调用

\item 
向类型添加参数

\item 
调用一个别名

\item 
实例化一个类型

\item 
实例化一个函数模板

\item 
使用SFINAE(替换失败不是错误)
\end{itemize}

为了演示该规则,看一下以下的代码:

\begin{lstlisting}[style=styleCXX]
template<bool>
struct conditional {
	template<typename T, typename F>
	using type = F;
};

template<>
struct conditional<true> {
	template<typename T, typename F>
	using type = T;
};

template<bool B, typename T, typename F>
using conditional_t = conditional<B>::template type<T, F>;
\end{lstlisting}

定义了一个条件模板别名,该别名存储一个类型,如果条件B为真,则解析为T,否则解析为F。编写这样一个实用程序的传统方法如下:

\begin{lstlisting}[style=styleCXX]
template<bool B, class T, class F>
struct conditional {
	using type = T;
};

template<class T, class F>
struct conditional<false, T, F> {
	using type = F;
};

template<bool B, class T, class F>
using conditional_t = conditional<B,T,F>::type;
\end{lstlisting}

但第二种方法的编译速度比第一种方法慢,因为它依赖于创建模板实例,而不是类型别名。

现在,来看看可以使用哪些工具及其特性,可以用来减少编译的时间。

\hspace*{\fill} \\ %插入空行
\noindent
\textbf{利用工具}

使用单个编译单元构建,或统一构建,可以使构建速度更快。它不会加快每个项目的速度,但若头文件中有大量的代码,可以试一试。统一构建是通过在一个转译单元中包含所有的\texttt{.cpp}文件来构建的。另一个类似的想法是使用预编译头文件。诸如CMake的Cotire插件可以使用这两种技术。CMake 3.16还增加了对统一构建的原生支持,可以为一个目标启用,\texttt{set\_target\_properties(<target> PROPERTIES UNITY\_BUILD ON)},或者通过设置\texttt{CMAKE\_UNITY\_BUILD}为true来启用全局统一构建。如果只是想要PCH(预编译头文件),只需要去看看CMake 3.16的\texttt{target\_precompile\_headers}就好。

\begin{tcolorbox}[colback=webgreen!5!white,colframe=webgreen!75!black, title=TIP]
\hspace*{0.7cm}若在C++文件中包含了太多的内容,可以考虑使用一个名为include-what-you-use的工具进行整理。可以将声明类型和函数放到头文件中,并且可以减少编译时间。
\end{tcolorbox}

如果项目需要很长时间才能链接,也有一些方法可以解决这个问题。使用不同的链接器,如LLVM的LLD或GNU的Gold,可以有很大的帮助,因为可以使用多线程链接。如果无法使用不同的链接器,可以尝试使用\texttt{-fvisibility-hidden}或\texttt{-fvisibility-inlineshidden}这样的标志,并在源代码中使用适当的注释,只标记希望在动态库中可见的函数。这样,链接器要做的工作就更少了。如果正在对链接时间进行优化,那么可以尝试只对性能关键的构建进行优化:计划分析的构建和那些用于生产的构建。否则,可能会浪费开发人员的时间。

If you're using CMake and aren't tied to a specific generator (for example, CLion requires using the Code::Blocks generator), you can replace the default Make generator with a faster one. Ninja is a great one to start with as it was created specifically to reduce build times. To use it, just pass -G Ninja when invoking CMake.

There are still two more great tools that will surely give you a boost. One of them is Ccache.It's a tool that runs its cache of C an d C++ compilation outputs. If you're trying to build the same thing twice, it will get the results from the cache instead of running the compilation. It keeps the statistics, such as cache hits and misses, can remember the warnings that it should emit when compiling a specific file, and has many configuration options that you can store in the ~/.ccache/ccache.conf file. To obtain its statistics, just run ccache -\,-show-stats.

The second tool is IceCC (or Icecream). It's a fork of distcc, essentially a tool to distribute your builds across hosts. With IceCC, it's easier to use a custom toolchain. It runs the iceccd daemon on each host and an icecc-scheduler service that manages the whole cluster. The scheduler, unlike in distcc, makes sure to only use the idle cycles on each machine, so you won't end up overloading other people's workstations.

To use both IceCC and Ccache for your CMake builds, just add -DCMAKE\_C\_COMPILER\_LAUNCHER="ccache;icecc" -DCMAKE\_CXX\_COMPILER\_LAUNCHER="ccache;icecc" to your CMake invocation. If you're compiling on Windows, instead of the last two tools, you could use clcache and Incredibuild or look for other alternatives.

Now that you know how to build fast, let's move on to another important topic.

\subsubsubsection{7.2.3\hspace{0.2cm}寻找代码的潜在问题}

Even the quickest builds aren't worth much if your code has bugs. There are dozens of flags to warn you of potential issues in your code. This section will try to answer which ones you should consider enabling.

First, let's start with a slightly different matter: how not to get warned about issues with code from other libraries. Getting warned about issues that you can't really fix isn't useful. Fortunately, there are compiler switches to disable such warnings. In GCC, for instance, you have two types of include files: regular (passed using -I) and system ones (passed using -isystem). If you specify a directory using the latter, you won't get warnings from the headers it contains. MSVC has an equivalent for -isystem: /external:I.Additionally, it has other flags to handle external includes, such as /external:anglebrackets, which tells the compiler to treat all files included using angle brackets as external ones, thus disabling warnings for them. You can specify a warning level for external files. You can also keep warnings coming from template instantiations caused by your code using /external:templates-. If you're looking for a portable way to mark include paths as system/external ones, and you're using CMake, you can add the SYSTEM keyword to a target\_include\_directories directive.

Speaking of portability, if you want to be conformant to a C++ standard (and you should), consider adding -pedantic to your compile options for GCC or Clang, or the /permissive- option for MSVC. This way, you'll get informed about every non-standard extension that you might be using. If you're using CMake, add the following line for each of your targets, set\_target\_properties(<target> PROPERTIES CXX\_EXTENSIONS OFF), to disable compiler-specific extensions.

If you're using MSVC, strive to compile your code with /W4, since it enables most of the important warnings. For GCC and Clang, try to use -Wall -Wextra -Wconversion -Wsign-conversion. The first one, despite its name, enables only some common warnings. The second, however, adds another bunch of warnings. The third one is based on the tips from a great book by Scott Meyers, titled Effective C++ (it's a set of good warnings, but check that it's not too noisy for your needs). The last two are about type conversions and signedness conversions. All those flags together create a sane safety net, but you can, of course, look for more flags to enable. Clang has a -Weverything flag. Try to periodically run a build with it to discover new, potential warnings that could be worth enabling in your codebase. You might be surprised at how many messages you get with this flag, although enabling some of the warning flags might not be worth the hassle. An MSVC alternative is named /Wall. Take a look at the following tables to see some other interesting options that are not enabled by the preceding 9:

GCC/Clang:
\begin{table}[H]
	\begin{tabular}{|l|l|}
		\hline
		\textbf{Flag}         & \textbf{Meaning}                                                                                                                                    \\ \hline
		-Wduplicated-cond     & \begin{tabular}[c]{@{}l@{}}Warn when the same condition is used in if and else-if\\ blocks.\end{tabular}                                            \\ \hline
		-Wduplicated-branches & Warn if both branches contain the same source code.                                                                                                 \\ \hline
		-Wlogical-op          & \begin{tabular}[c]{@{}l@{}}Warn when operands in logical operations are the same and\\ when a bitwise operator should be used instead.\end{tabular} \\ \hline
		-Wnon-virtual-dtor    & \begin{tabular}[c]{@{}l@{}}Warn when a class has virtual functions but not a virtual\\ destructor.\end{tabular}                                     \\ \hline
		-Wnull-dereference    & \begin{tabular}[c]{@{}l@{}}Warn about null dereferences. This check may be inactive in\\ unoptimized builds.\end{tabular}                           \\ \hline
		-Wuseless-cast        & Warn when casting to the same type.                                                                                                                 \\ \hline
		-Wshadow              & \begin{tabular}[c]{@{}l@{}}A whole family of warnings about declarations that shadow\\ other, previous declarations.\end{tabular}                   \\ \hline
	\end{tabular}
\end{table}

MSVC:

\begin{table}[H]
	\begin{tabular}{|l|l|}
		\hline
		\textbf{Flag} & \textbf{Meaning}                                      \\ \hline
		/w44640       & Warn on non-thread-safe static member initialization. \\ \hline
	\end{tabular}
\end{table}

One last thing worth mentioning is the question: to -Werror (or /WX on MSVC) or not to - Werror? This really depends on your personal preferences as issuing errors instead of warnings has its pros and cons. On the plus side, you won't let any of your enabled warnings slip by. Your CI build will fail and your code won't compile. When running multi-threaded builds, you won't lose any warnings in the quickly passing compilation messages. However, there are some minuses too. You won't be able to upgrade your compiler if it enables any new warnings or just detects more issues. The same goes for dependencies, which can deprecate some functions they provide. You won't be able to deprecate anything in your code if it's used by other parts of your project. Fortunately, you can always use a mixed solution: strive to compile with -Werror, but disable it when you need to do the things it inhibits. This requires discipline, as if any new warnings are to slip in, you may have a hard time eliminating them.

\subsubsubsection{7.2.4\hspace{0.2cm}使用以编译器为中心的工具}

Nowadays, compilers allow you to do much more with them than a few years back. This is owing to the introduction of LLVM and Clang. By providing APIs and a modular architecture allowing easy reuse, caused tools such as sanitizers, automatic refactoring, or code completion engines to flourish. You should consider taking advantage of what this compiler infrastructure offers you. Use clang-format to ensure all the code in your code base conforms to a given standard. Consider adding pre-commit hooks using the precommit tool to reformat new code before commit. You can also add Python and CMake formatters to the mix. Statically analyze the code using clang-tidy – a tool that actually understands your code instead of just reasoning about it. There's a ton of different checks this tool can perform for you, so be sure to customize the list and options to your specific needs. You can also run nightly or weekly tests of your software with sanitizers enabled. This way, you can detect threading issues, undefined behavior, memory access, management issues, and more. Running tests using debug builds can also be of value if your release builds have assertions disabled.

If you think that more could be done, you can consider writing your own code refactorizations using Clang's infrastructure. There's already a clang-rename tool if you want to see how to create an LLVM-based tool of your own. Additional checks and fix-its for clang-tidy are also not that hard to create, and they can save you hours of manual labor.

You can integrate many tools into your building process. Let's now discuss the heart of the process: the build system.












