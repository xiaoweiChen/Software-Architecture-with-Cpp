
编译器是每个开发者工作中最重要的工具之一。本节中,将介绍一些有用的技巧。这只是冰山一角,因为关于这些工具及其各种可用标志、优化、功能和其他细节汇总起来就是可以成书的量级,GCC还有有一个专门的Wiki页面,里面有关于编译器的书籍列表!可以在本章末尾的扩展阅读中找到它。

\subsubsubsection{7.2.1\hspace{0.2cm}多编译器}

构建过程中,需要考虑使用多个编译器(而非一个),原因是它们可以检测代码中的不同问题。例如,MSVC默认启用签名检查。使用多个编译器可以提前发现可能遇到的可移植性问题,特别是决定在不同的操作系统上编译代码时,例如:从Linux迁移到Windows或其他操作系统。要无痛完成这些工作,需要编写可移植的、符合ISO C++的代码。Clang是一个比较好的选择,其力求比GCC更符合C++标准。如果正在使用MSVC,可以试着添加\texttt{/permissive-}选项(在Visual Studio 17加入,对于使用15.5+版本创建的项目默认启用)。对于GCC,在为代码选择C++标准时尽量不要使用GNU扩展(例如,使用\texttt{-std=c++17}而不是\texttt{-std=gnu++17})。如果目标是性能,那么使用各种编译器来构建也可以为特定的用例,选择运行最快的二进制文件。

\begin{tcolorbox}[colback=webgreen!5!white,colframe=webgreen!75!black, title=TIP]
\hspace*{0.7cm}无论为发布版本选择哪种编译器,都可以考虑使用Clang进行开发。它可以在macOS、Linux和Windows上运行,支持与GCC相同的标志集,旨在提供最快的构建时间和简明的编译错误。
\end{tcolorbox}

如果正在使用CMake,有两种常见的方法来添加另一个编译器。一种是在调用CMake时传递相应的编译器,像这样:

\begin{tcblisting}{commandshell={}}
mkdir build-release-gcc
cd build-release-gcc
cmake .. -DCMAKE_BUILD_TYPE=Release -DCMAKE_C_COMPILER=/usr/bin/gcc \
  -DCMAKE_CXX_COMPILER=/usr/bin/g++
\end{tcblisting}

也可以在调用CMake之前设置\texttt{CC}和\texttt{CXX},但这些环境变量在不同的平台(如macOS)上会有所不同。

另一种方法是使用工具链文件。如果只使用不同的编译器,这可能没必要,但若想要交叉编译时,这是首选的解决方案。要使用一个工具链文件,应该作为一个参数传递给CMake:\texttt{-DCMAKE\_TOOLCHAIN\_FILE=toolchain.cmake}。

\subsubsubsection{7.2.2\hspace{0.2cm}减少构建时间}

程序员每年都要花费无数个小时等待他们的构建完成。减少构建时间是提高整个团队生产力的一种简单方法,所以这里讨论一些实现它的方法。

\hspace*{\fill} \\ %插入空行
\noindent
\textbf{快速编译器}

提高构建速度的最简单方法之一就是升级编译器。例如,通过将Clang升级到7.0.0,可以使用预编译头(PCH)文件节省多达30\%的构建时间。从Clang 9开始,就有了\texttt{-ftime-trace}选项,该选项可以提供有关它处理的所有文件的编译时间的信息。其他编译器也有类似的开关:查看GCC的\texttt{-ftime}报告或MSVC的/Bt和\texttt{/d2cgsummary}。通常可以通过切换编译器来获得更快的编译速度,这在开发机器上特别有用,例如:Clang编译代码的速度通常比GCC快。

当有了一个快速的编译器后,再来看看需要编译什么。

\hspace*{\fill} \\ %插入空行
\noindent
\textbf{反思模板}

编译过程的不同部分需要不的时间来完成,这对于编译时构造尤其重要。Odin Holmes的实习生之一Chiel Douwes在对各种模板操作的编译时成本进行基准测试的基础上创建了Chiel规则。其他基于类型的模板元编程技巧,可以在Odin Holmes的《基于类型的模板元编程并未消亡》讲座中看到。从最快到最慢,排列如下:

\begin{itemize}
\item 
查找已实例化的类型(例如,模板实例化)

\item 
将参数传递给别名调用

\item 
向类型添加参数

\item 
调用一个别名

\item 
实例化一个类型

\item 
实例化一个函数模板

\item 
使用SFINAE(替换失败不是错误)
\end{itemize}

为了演示该规则,看一下以下的代码:

\begin{lstlisting}[style=styleCXX]
template<bool>
struct conditional {
	template<typename T, typename F>
	using type = F;
};

template<>
struct conditional<true> {
	template<typename T, typename F>
	using type = T;
};

template<bool B, typename T, typename F>
using conditional_t = conditional<B>::template type<T, F>;
\end{lstlisting}

代码定义条件模板别名,该别名存储一个类型,如果条件B为真,则解析为T,否则解析为F。编写这样一个实用程序的传统方法如下:

\begin{lstlisting}[style=styleCXX]
template<bool B, class T, class F>
struct conditional {
	using type = T;
};

template<class T, class F>
struct conditional<false, T, F> {
	using type = F;
};

template<bool B, class T, class F>
using conditional_t = conditional<B,T,F>::type;
\end{lstlisting}

第二种方法的编译速度比第一种方法慢,因为它依赖于创建模板实例,而不是类型别名。

现在,来看看可以使用哪些工具及其特性,可以用来减少编译的时间。

\hspace*{\fill} \\ %插入空行
\noindent
\textbf{利用工具}

使用单个编译单元构建,或统一构建,可以使构建速度更快。它不会加快每个项目的速度,但若头文件中有大量的代码,可以试一试。统一构建是通过在一个转译单元中包含所有的\texttt{.cpp}文件来构建的,另一个类似的想法是使用预编译头文件。诸如CMake的Cotire插件可以使用这两种技术。CMake 3.16还增加了对统一构建的原生支持,可以为一个目标启用,\texttt{set\_target\_properties(<target> PROPERTIES UNITY\_BUILD ON)},或者通过设置\texttt{CMAKE\_UNITY\_BUILD}为true来启用全局统一构建。如果只是想要PCH(预编译头文件),只需要去看看CMake 3.16的\texttt{target\_precompile\_headers}就好。

\begin{tcolorbox}[colback=webgreen!5!white,colframe=webgreen!75!black, title=TIP]
\hspace*{0.7cm}若在C++文件中包含了太多的内容,可以考虑使用一个名为include-what-you-use的工具进行整理。可以将声明类型和函数放到头文件中,并且可以减少编译时间。
\end{tcolorbox}

如果项目需要很长时间才能链接,也有一些方法可以解决这个问题。使用不同的链接器,如LLVM的LLD或GNU的Gold,可以有很大的帮助,因为可以使用多线程链接。如果无法使用不同的链接器,可以尝试使用\texttt{-fvisibility-hidden}或\texttt{-fvisibility-inlineshidden}这样的标志,并在源代码中使用适当的注释,只标记希望在动态库中可见的函数。这样,链接器要做的工作就更少了。如果正在对链接时间进行优化,可以尝试只对性能关键的构建进行优化:计划分析的构建和那些用于生产的构建。

如果正在使用CMake,并且没有绑定到特定的生成器(例如,CLion需要使用Code::Blocks生成器),可以用更快的生成器替换默认的Make生成器。Ninja是一个很好的开始,它专为减少构建时间而创建的,只需在调用CMake时传递\texttt{-G Ninja}就可以使用。

还有两个很棒的工具,其中一个就是Ccache。它是一个执行C和C++编译输出缓存的工具。如果试图构建相同的内容两次,将从缓存中获得结果,而不是运行编译。它保留统计数据,例如:缓存命中和未命中,可以记住编译特定文件时应该发出的警告,并且有许多配置选项可以存储在\texttt{~/.ccache/ccache.conf}文件中。要获取它的统计信息,只需运行\texttt{ccache -\,-show-stats}即可。

第二个工具是IceCC(或Icecream)。它是distcc的一个分支,本质上是一个跨主机分发构建的工具。有了IceCC,使用自定义工具链就更容易了。它在每个主机上运行iceccd守护进程,并运行一个icecc-scheduler服务来管理整个集群。与distcc不同的是,调度程序确保只使用每台机器上的空闲周期,因此不会导致其他人的工作机超载。

要使用IceCC和Ccache为您的CMake构建,只需在CMake调用中添加:

\begin{tcblisting}{commandshell={}}
-DCMAKE_C_COMPILER_LAUNCHER="ccache;icecc"
-DCMAKE_CXX_COMPILER_LAUNCHER="ccache;icecc"
\end{tcblisting}

如果在Windows上编译,想使用其他类似的工具,可以选择使用clcache和Incredibuild或其他替代工具。

\subsubsubsection{7.2.3\hspace{0.2cm}寻找代码的潜在问题}

如果代码有bug,即使是最快的构建也没有多大价值。有许多标志可以用来警告代码中可能存在的问题。本节将展示应该考虑启用哪些选项。

首先,让从一个稍微不同的问题开始:如何不收到来自其他库的代码问题的警告。对于无法解决的问题上的警告没有任何价值。幸运的是,有一些编译器开关可以禁用此类警告。例如,在GCC中有两种类型的包含文件:常规文件(使用\texttt{-I}传递)和系统文件(使用\texttt{-system}传递)。如果使用后者指定目录,则不会从相应的头文件中获得警告。MSVC有一个\texttt{-system}的等价选项:\texttt{/external:I}。此外,它还有其他标志来处理外部包含,比如\texttt{/external:anglebrackets},它告诉编译器将使用尖括号包含的所有文件视为外部文件,从而禁用对它们的警告。可以为外部文件指定警告级别,也可以让警告来自于使用\texttt{/external}进行模板实例化的代码:如果正在寻找一种可移植的方式来将包含路径标记为系统/外部路径,并且正在使用CMake,可以将\texttt{SYSTEM}关键字添加到\texttt{target\_include\_directories}指令的参数中。

说到可移植性,如果想要符合C++标准,考虑在GCC或Clang的编译选项中加入\texttt{-pedantic},或者在MSVC中加入\texttt{/permissive-}选项。这样,就会了解每个可能使用非标准扩展的模块。如果正在使用CMake,为每个目标添加设置相应的属性,\texttt{set\_target\_properties(<target> PROPERTIES CXX\_EXTENSIONS OFF)}以禁用特定于编译器的扩展。

如果正在使用MSVC,可以尝试使用\texttt{/W4}来编译代码,因为它启用了大多数重要的警告。对于GCC和Clang,尝试使用\texttt{-Wall -Wextra -Wconversion -Wsign-conversion}。第一个,只启用一些常见的警告。然而,第二个又增加了一堆警告。第三个是基于Scott Meyers的著作《Effective C++》中的提示(这是一组很好的警告,但是要确保是否符合需求)。后两个是关于类型转换和签名转换的。所有这些标志创建了一个完整的安全网,当然可以启用更多的标志。Clang有一个\texttt{-Weverything}标志,尝试使用它运行构建,可以发现新的警告。虽然启用一些警告标志可能没这么麻烦,但是开发者可能会对使用此标志获得的消息数量感到惊讶。MSVC的一个替代方案命名为\texttt{/Wall}。看看下面的表格,了解一下其他没有启用的有趣选项:

GCC/Clang:
\begin{table}[H]
	\begin{tabular}{|l|l|}
		\hline
		\textbf{标志}         & \textbf{含义}                                                                                                                                    \\ \hline
		-Wduplicated-cond     & \begin{tabular}[c]{@{}l@{}}当在if和else-if块中使用相同的条件时发出警告\end{tabular}                                            \\ \hline
		-Wduplicated-branches & 如果两个分支包含相同的源代码,则发出警告。                                                                                                 \\ \hline
		-Wlogical-op          & \begin{tabular}[c]{@{}l@{}}当逻辑操作中的操作数相同且应该使用位操作符时发出警告。\end{tabular} \\ \hline
		-Wnon-virtual-dtor    & \begin{tabular}[c]{@{}l@{}}当类有虚函数,但没有虚析构函数时发出警告。\end{tabular}                                     \\ \hline
		-Wnull-dereference    & \begin{tabular}[c]{@{}l@{}}警告空解引用。在未优化的构建中,此检查可能是不开启的。\end{tabular}                           \\ \hline
		-Wuseless-cast        & 转换为相同类型时发出警告。                                                                                                                 \\ \hline
		-Wshadow              & \begin{tabular}[c]{@{}l@{}}所有关于覆盖了之前声明的警告。\end{tabular}                   \\ \hline
	\end{tabular}
\end{table}

MSVC:

\begin{table}[H]
	\begin{tabular}{|l|l|}
		\hline
		\textbf{Flag} & \textbf{Meaning}                                      \\ \hline
		/w44640       & 对非线程安全的静态成员初始化发出警告。 \\ \hline
	\end{tabular}
\end{table}

最后一个问题:何时使用\texttt{-Werror}(或在MSVC是\texttt{/WX})或不使用\texttt{-Werror}?这实际上取决于个人偏好,因为发出错误而不是警告有其优缺点。从积极的方面来说,不会让警告溜走。CI构建将失败,代码将无法编译。在运行多线程构建时,不会在快速传递的编译消息中丢失警告。然而,也有一些缺点。如果编译器启用了新的警告或只是检测到更多问题,那么就无法升级编译器。依赖关系也是如此,依赖关系会使一些函数不受欢迎。如果代码中的内容在项目的其他部分使用,就不能弃用它。幸运的是,可以使用混合解决方案:使用\texttt{-Werror}进行编译,但在需要时禁用它。这就需要代码的编写很有纪律性,如果有新的警告出现,可能很难消除。

\subsubsubsection{7.2.4\hspace{0.2cm}以编译器为中心的工具}

与几年前相比,编译器可以做更多的事情,这是由于LLVM和Clang的引入。通过提供API和重用的模块化架构,使消毒工具、自动重构或代码完成引擎等工具蓬勃发展,并可以利用这个编译器基础设施提供的优势。使用clang-format来确保代码库中的所有代码都符合给定的标准。考虑添加预提交钩子,使用预提交工具在提交之前重新格式化新代码。也可以添加Python和CMake格式器。使用clang-tidy静态分析代码——这个工具能够理解代码,而不仅仅是对代码进行推演。这个工具可以执行不同的检查,所以一定要根据特定需求定制列表和选项。还可以在启用杀毒软件的情况下,每晚或每周运行软件测试。通过这种方式,可以检测线程问题、未定义的行为、内存访问、管理问题等等。如果发布版本禁用了断言,那么使用调试版本运行测试也可以。

如果认为还可以做得更多,可以考虑使用Clang的基础架构编写自己的代码重构。如果想了解如何创建自己的基于llvm的工具,那么可以参考clang-rename工具。对clang-tidy进行额外的检查和修复也不是那么难,这个工具可以节省几个小时的体力劳动。

可以将许多工具集成到构建过程中。现在就讨论这个过程的核心:构建系统。












