In this chapter, you've learned a lot about building and packaging your code. You're now able to write faster-building template code, know how to choose the tools to compile your code faster (you'll learn more about tooling in the next chapter), and know when to use forward declarations instead of \#include directives.

Aside from that, you can now define your build targets and test suites using Modern CMake, manage external dependencies using find modules and FetchContent, create packages and installers in various formats, and last but not least, use Conan to install your dependencies and create your own artifacts.

In the next chapter, we will look at how to write code that would be easy to test. Continuous integration and continuous deployment are useful only if you have good test coverage. Continuous deployment without comprehensive testing will allow you to introduce new bugs to production much quicker. This is not our goal when we design software architecture.