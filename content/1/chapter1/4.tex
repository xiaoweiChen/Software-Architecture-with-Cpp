
区分架构的好与坏非常重要,但也并不是那么容易。识别反模式显得尤为重要,但是对于好的架构来说,必须从软件中获得相应的交付期望,无论是关于功能需求、解决方案的属性,还是处理不同的限制。其中许多方案都可以很容易地从架构的上下文中派生出来。

\subsubsubsection{1.4.1\hspace{0.2cm}架构的上下文}

上下文是架构师在设计可靠的解决方案时要考虑的因素,包括需求、假设和限制,可以来自利益相关方,也可以来自业务和技术环境。还会影响利益相关方和环境,例如:公司需要涉猎一个新的细分市场。

\subsubsubsection{1.4.2\hspace{0.2cm}利益相关方}

利益相关方是所有与产品的人,这些人可以是客户、系统用户或管理人员。沟通是每个架构师的关键技能,适当地统筹各方需求,其关键在于要以各方预期的交付方式进行交付,并满足交付各方的期望。

这对利益相关群体来说很重要,所以尽量收集所有相关群体的意见。

客户可能会关心编写和运行软件的成本、交付的功能、生命周期、上市时间,以及解决方案的质量。

系统的用户可以分为两组:用户和管理员。前者通常关心诸如软件的可用性、用户体验和性能的事情。后者更关心的是用户管理、系统配置、安全性、备份和恢复。

最后,对于从事管理工作的群体来说,保持低成本开发十分重要,并且还要实现业务目标,在按照开发进度进行的同时,还要保持产品的质量。

\subsubsubsection{1.4.3\hspace{0.2cm}业务和技术环境}

架构可能会受到公司业务的影响,可能的原因有上市时间、时间表、组织结构、劳动力使用情况,以及对现有资产的配比。

所谓技术环境,指的是公司中使用的技术,或者出于需要成为解决方案的技术。需要集成的系统,也是技术环境的重要组成部分。软件工程师的技术专长在这里也很重要,架构师做出的技术决策可以影响项目的人员配置,所以初级开发人员与高级开发人员的比例会对项目管理有一定的影响。因此,好的架构应该把这些都考虑进去。

了解了这些,现在来讨论一个有争议的话题。作为架构师,这个话题在日常工作中很可能会遇到。















