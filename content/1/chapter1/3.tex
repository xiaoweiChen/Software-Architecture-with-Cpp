
Actually, a better question would be: why is caring about your architecture important? As we mentioned earlier, regardless of whether you put conscious effort into building it or not, you will end up with some kind of architecture. If after several months or even years of development you still want your software to retain its qualities, you need to take some steps earlier in the process. If you won't think about your architecture, chances are it won't ever present the required qualities.

So, in order for your product to meet the business requirements and attributes such as performance, maintainability, scalability, or others, you need to take care of its architecture, and it is best if you do it as early as you can in the process. Let's now discuss two things that each good architect wants to protect their projects from.


\subsubsubsection{1.3.1\hspace{0.2cm}Software decay}

Even after you did the initial work and had a specific architecture in mind, you need to continuously monitor how the system evolves and whether it still aligns with its users' needs, as those may also change during the development and lifetime of your software. Software decay, sometimes also called erosion, occurs when the implementation decisions don't correspond to the planned architecture. All such differences should be treated as technical debt.


\subsubsubsection{1.3.2\hspace{0.2cm}Accidental architecture}

Failing to track if the development adheres to the chosen architecture or failing to intentionally plan how the architecture should look will often result in a so-called accidental architecture, and it can happen regardless of applying best practices in other areas, such as testing or having any specific development culture. 

There are several anti-patterns that suggest your architecture is accidental. Code resembling a big ball of mud is the most obvious one. Having god objects is another important sign of this. Generally speaking, if your software is getting tightly coupled, perhaps with circular dependencies, but wasn't like that in the first place, it's an important signal to put more conscious effort into how the architecture looks.

Let's now describe what an architect must understand to deliver a viable solution.










