
在创建软件系统时,应该不断地问自己,所做的是否是客户所需要的。很多时候,客户不知道如何才能能满足他们的需求。成功架构师的角色是发现产品的需求,并确保需求得到满足。这里需要考虑三种不同类型的需求:功能需求、质量属性和约束条件。

\subsubsubsection{3.2.1\hspace{0.2cm}功能需求}

第一组是功能需求。这些定义了系统应该做什么,或者它应该提供什么功能。

\begin{tcolorbox}[colback=webgreen!5!white,colframe=webgreen!75!black, title=TIP]
\hspace*{0.7cm}请记住,功能并不总是影响体系结构,因此必须注意哪些需求将影响解决方案。
\end{tcolorbox}

通常,如果功能需求具有某些必须满足的特性,那么在架构上就很重要。考虑为参加多米尼加博览会(Dominican Fair)的商人和游客开发一款应用程序,该博览会是在Gdańsk举办的年度活动,内容包括音乐、各种艺术和商店。下面是一些功能需求的例子:

\begin{itemize}
\item 
\textit{作为一个店主,我想过滤包含特定产品的订单。}

\item 
\textit{单击“订阅”按钮,将客户添加到选定商家的通知列表中。}
\end{itemize}

第一个要求,必须有一个具有搜索功能的跟踪订单和产品的组件。根据UI的显示方式和应用的规模,可以只在应用中添加一个简单的页面,或者可能需要Lucene(一个全文搜索引擎)或Elasticsearch(搜索引擎解决方案)等特性。这意味着个\textbf{体系结构重要需求(ASR)},可以影响到架构。

The second example is even more straightforward; now we know we need to have a service for subscribing and sending notifications. This is definitely an architecturally significant functional requirement. Let's now look at some \textbf{Non-Functional Requirements (NFRs)} that can also be ASRs.

By the way, the first requirement is actually given as a user story. User stories are requirements given in the following format: "\textit{As a <role>, I can/want to <capability>, so that <benefit>.}" This is a common way to phrase requirements and can help stakeholders and developers find common ground and communicate better.

\subsubsubsection{3.2.2\hspace{0.2cm}非功能需求}

Instead of focusing on what functionality your system should have, nonfunctional requirements focus on how well and under which conditions the system should perform said functionality. This group consists of two main subgroups: \textbf{Quality Attributes (QAs)} and \textbf{constraints}.

\hspace*{\fill} \\ %插入空行
\noindent
\textbf{质量属性}

\textbf{质量属性(QAs)} are the traits of your solution, such as performance, maintainability, and user-friendliness. There are dozens, if not hundreds, of different qualities your software can have. Try to focus just on the important ones instead of listing all that come to your mind when choosing which ones your software should have. Examples of quality attribute requirements include the following:

\begin{itemize}
\item 
The system will respond in under 500 ms for 99.9\% of all requests under usual load (don't forget to specify what the usual load is or will be).

\item 
The website will not store customer credit card data used in the payment process (an example of confidentiality).

\item
When updating the system, if updating any component fails, the system will be rolled back to a state prior to the update (survivability).

\item 
As a user of Windows, macOS, and Android, I want to be able to use the system from all of them (portability; try to understand whether it's needed to support platforms such as desktop, mobile, and/or web).
\end{itemize}

While catching functional requirements in a backlog is pretty straightforward, we cannot say the same regarding quality attribute requirements. Fortunately, there are a few ways you could approach this:

\begin{itemize}
\item 
Some of them can be expressed in the \textbf{Definition of done} or \textbf{Acceptance criteria} for your tasks, stories, and releases.

\item 
Others can be expressed directly as user stories, as shown in the last example previously.

\item 
You can also check them as part of design and code reviews and create automated tests for some of them.
\end{itemize}

\hspace*{\fill} \\ %插入空行
\noindent
\textbf{约束条件}

Constraints are the non-negotiable decisions that you must follow while delivering the project. Those can be design decisions, technological ones, or even political (regarding people or organizational matters). Two other common constraints are \textbf{time} and \textbf{budget}. Examples of constraints could be as follows:

\begin{itemize}
\item 
\textit{The team will never grow beyond four developers, one QA engineer, and one sysadmin.}

\item 
\textit{Since our company leverages Oracle DB in all its current products, the new product must use it too so we can make the most of our expertise.
}
\end{itemize}

nonfunctional requirements are always going to influence your architecture. It's essential not to over-specify them, as having false positives will be a constant burden during product development. It's equally important to not under-specify them as this can later come out in missed sales opportunities or failing to comply with regulatory bodies' requirements.

In the next section, you will learn how to strike a balance between those two extremes and to focus on just those requirements that really matter in your specific case.





