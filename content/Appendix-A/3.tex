While related to cloud-native design, serverless architecture is a popular topic on its own. It gained a lot of popularity since the introduction of FaaS or CaaS products, such as AWS Lambda, AWS Fargate, Google Cloud Run, and Azure Functions.

Serverless is mostly an evolution of PaaS products such as Heroku. It abstracts the underlying infrastructure so that developers can focus on the application and not on infrastructural choices.

An additional benefit of serverless over older PaaS solutions is that you don't have to pay for what you don't use. Rather than paying for a given service level, you typically pay for the actual execution time of the deployed workload with serverless. If you only want to run a given piece of code once a day, you don't need to pay a monthly fee for an underlying server.

While we didn't get into too much detail about serverless, it is rarely used with C++. When it comes to FaaS, only AWS Lambda currently supports C++ as a possible language. Since containers are language-agnostic, you can use C++ applications and functions with CaaS products such as AWS Fargate, Azure Container Instances, or Google Cloud Run.

Serverless functions may still be relevant to you if you want to run non-C++ auxiliary code  used along with your C++ application. Maintenance tasks and scheduled jobs are an excellent fit for serverless and they usually don't require the performance or efficiency of C++ binaries.



