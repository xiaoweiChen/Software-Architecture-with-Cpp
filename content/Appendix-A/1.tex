现在让我们讨论应用程序的存储。首先来决定是使用SQL、NoSQL还是其他什么。

一个经验法则是,根据数据库的大小来决定技术。对于小型数据库,例如那些大小永远不会增长到TB的数据库,使用SQL是一种有效的方法。如果有一个非常小的数据库或想要创建一个内存缓存,可以尝试SQLite。如果计划使用1个TB,并且保证不会超过这个值,那么最好的选择是使用NoSQL。某些情况下,仍然可以使用SQL数据库,但是由于硬件成本的原因,很快就会变得昂贵,因为需要为主节点使用一个庞大的服务器。即使这不是问题,也应该衡量性能是否足以满足需求,并为长期维护做好准备。某些情况下,使用Citus(本质上是一个分片的PostgreSQL)等技术来运行SQL机器集群也比较适合。但在这种情况下,通常使用NoSQL更便宜、更简单。如果数据库的大小超过10TB,或者需要实时获取数据,请考虑使用数据仓库而不是NoSQL。