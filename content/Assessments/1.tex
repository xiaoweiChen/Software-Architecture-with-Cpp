

\begin{enumerate}
\item 
Why should you care about software architecture?

\begin{itemize}
\item 
Architecture allows you to achieve and maintain the requisite qualities of software. Being mindful and caring about it prevents a project from having accidental architecture, thereby losing quality, and also prevents software decay.
\end{itemize}

\item 
Should the architect be the ultimate decision maker in an Agile team?
\begin{itemize}
\item 
No. Agile is about empowering the whole team. An architect brings their experience and knowledge to the table, but if a decision has to be accepted by the whole team, the team should own it, not just the architect. Considering the needs of stakeholders is also of great importance here.
\end{itemize}

\item 
How does the Single Responsibility Principle (SRP) relate to cohesion?

\begin{itemize}
\item 
Following the SRP leads to better cohesion. If a component starts having multiple responsibilities, usually it becomes less cohesive. In such instances, it's best to just refactor it into multiple components, each having a single responsibility. This way, we increase cohesiveness, so the code becomes easier to understand, develop, and maintain.
\end{itemize}

\item
During what phases of a project's lifetime can benefit be derived from having an architect?

\begin{itemize}
\item 
An architect can bring value to a project from its inception until the time it goes into maintenance. The most value can be achieved during the early phases of the project's development, as this is where key decisions about how it should look will be taken. However, this doesn't mean that architects cannot be valuable during development. They can keep the project on the right course and on track. By aiding decisions and overseeing the project, they ensure that the code doesn't end up with accidental architecture and is not subject to software decay.
\end{itemize}

\item
What's the benefit of following the SRP?

\begin{itemize}
\item 
Code that follows the SRP is easier to understand and maintain. This also means that it has fewer bugs.
\end{itemize}

\end{enumerate}