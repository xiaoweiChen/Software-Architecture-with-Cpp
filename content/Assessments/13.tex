\begin{enumerate}
\item
为什么微服务可以更好地利用系统资源?
\begin{itemize}
\item 
只扩展缺少的资源,比扩展整个系统要容易得多。
\end{itemize}

\item
微服务和单应用如何共存(在一个不断发展的系统中)?
\begin{itemize}
\item 
新功能可能会作为微服务开发,而一些功能可能会从整体中分离出来并外包出去。
\end{itemize}

\item
哪种类型的团队从微服务中受益最大?
\begin{itemize}
\item 
遵循DevOps原则的跨职能自治团队。
\end{itemize}

\item
为什么在引入微服务时需要一个成熟的DevOps方法?
\begin{itemize}
\item 
测试和部署大量的微服务几乎不可能由单独的团队手工完成。
\end{itemize}

\item
什么是统一日志记录层?
\begin{itemize}
\item 
是一种可配置的工具,用于收集、处理和存储日志。
\end{itemize}

\item
日志记录和跟踪有什么不同?
\begin{itemize}
\item 
日志记录通常是人类可读的,关注于操作;而跟踪通常是机器可读的,关注于调试。
\end{itemize}

\item
为什么REST不是连接微服务的最佳选择
\begin{itemize}
\item 
例如与gRPC相比,可能提供更大的开销。
\end{itemize}

\item
微服务的部署策略是什么?各自的好处是什么?
\begin{itemize}
\item 
每个主机提供单个服务——更容易根据工作负载调整机器。

\item 
每个主机提供多个服务——更好地利用资源。
\end{itemize}
\end{enumerate}